\chapter{Ultrafiolett stråling}
Ultrafiolett stråling har for lite energi-rike fotoner til å ionisere atomer, men kan likevel gjøre skade på kroppsvev. Den viktigste skademekanismen er produksjon av tymin-dimerer i DNA-molekylene våre \cite{tve-uv}. Tymin-dimerer er sammenkobling av to tymin-baser som er plassert ved siden av hverandre i DNA-strengen. Produksjon av tymin-dimerer hindrer replisering av DNA-strengen. Grunnen til at UV-lys kan forårsake slike skader i DNA'et er at basene fra hver side av DNA-molekylet er bundet sammen med hydrogenbindinger som det kreves relativt lite energi for å bryte.