\chapter{Skalarer og vektorer}
En del fysiske størrelser kan spesifiseres fullstendig ved hjelp av et tall + enhet. Dette er en skalar. Merk at den skalare størrelsen kan ha samme verdi på alle punkter i rommet (eksempel: lysets hastighet i vakuum) eller variere fra sted til sted (eksempel: temperatur). Skalare størrelser kan også endre seg med tiden. Når vi regner med skalarer kan vi bruke alle de vanlige regneartene.

Andre fysiske størrelser er ikke fullstendig spesifisert ved hjelp av bare et tall + enhet. For eksempel holder det ikke å si at det blåser $10~\mathrm{m/s}$ -- vi vil som regel vite retningen også. En størrelse som må spesifiseres både ved hjelp av verdi og retning kalles en vektor. 


\section{Vektoroperasjoner}
Akkurat som skalarer kan vektorer ha samme verdi og retning i alle punkter i rommet, eller verdi og/eller retning kan variere fra punkt til punkt. Verdi og retning kan også endre seg med tiden. For vektorer har vi i tillegg til addisjon og subtraksjon tre typer multiplikasjon:
\begin{itemize}
\item Multiplikasjon med en skalar $\to$ gir en vektor som svar
\item Indreprodukt/prikkprodukt mellom to vektorer $\to$ gir en skalar som svar
\item Vektorprodukt/kryssprodukt mellom to vektorer $\to$ gir en vektor som svar
\end{itemize}

\section{Enhetsvektorer}
Når vi jobber med vektorstørrelser er det ofte nyttig å introdusere en eller flere \emph{enhetsvektorer}. En enhetsvektor er ganske enkelt en vektor med lengde 1, og nytten av den er at den peker i en bestemt retning. Mer at selv når vi jobber med vektorfelt som har dimensjon (f.eks. m/s eller N/C) er enhetsvektoren dimensjonsløs. Vanligvis bruker vi enhetsvektorer som peker i retningene til aksene i koordinatsystemet vårt, f.eks. i $x$-, $y$-, og $z$-retning. Dette blir diskutert i mer detalj nedenfor. Noen ganger ønsker vi imidlertid å bruke en enhetsvektor som peker i samme retning som en vilkårlig vektor. Hvis vi for eksempel har en posisjonsvektor $\vec{r}$ kan det noen ganger være nyttig å skrive den som 
\begin{equation}
	\vec{r} = r \hat{r}
	\label{eq:vektorer:r}
\end{equation}
der $r$ er lengden av vektoren, altså $r = |\vec{r}|$ og $\hat{r}$ er en enhetsvektor som peker i samme retning som $\vec{r}$. Ved å se på ligning (\ref{eq:vektorer:r}) finner vi ut at enhetsvektoren kan finnes fra vectoren $\vec{r}$ som
\begin{displaymath}
	\hat{r} = \frac{\vec{r}}{|\vec{r}|}.
\end{displaymath}

\subsection{Enhetsvektorer i et kartesisk koordinatsystem}
Vi ser nå på et vanlig kartesisk koordinatsystem, altså et koordinatsystem som består av en $x$-, $y-$ og $z$-akse som står normalt på hverandre. Vi vet da at enhver vektor kan dekomponeres i $x$-, $y$- og $z$-komponenter. For eksempel kan vektoren $\vec{r}$ skrives som $(r_x, r_y, r_z)$ der vi har listet opp hver enkelt komponent. Ofte er det nyttigere å bruke en notasjon der vi multipliserer hver komponent med enhetsvektoren som går i den aktuelle retningen:
\begin{displaymath}
\begin{aligned}
	\vec{r} &= r_x\hat{\i} + r_y\hat{\j} + r_z\hat{k} \\
	&= r_x\ex + r_y\ey + r_z\ez.
\end{aligned}
\end{displaymath}
Den eneste forskjellen på de to linjene her er at det er brukt uliknotasjon for enhetsvektorene. Begge notasjonene er i vanlig bruk, så det er nødvendig å kjennet til begge. I denne teksten velger jeg å bruke $\ex, \ey, \ez$ fordi notasjonen viser eksplisitt hvilken retning hver enhetsvektor peker, og fordi den er helt tilsvarende den mest brukte notasjonen for sylinder- og sfæriske koordinater.

Det er nyttig å kjenne resultatet av prikk- og kryssprodukt mellom de ulike enhetsvektorene, fordi det er da enkelt å regne ut prikk- eller kryssprodukt mellom to vilkårlige vektorer som er uttrykt ved hjelp av enhetsvektorene. Siden alle enhetsvektorene har lengde 1 og står normalt på hverandre er prikk-produktet svært enkelt:
\begin{displaymath}
	\hat{e}_i \cdot \hat{e}_j = 
	\left\{
		\begin{aligned}
		1 \text{ hvis } i = j \\
		0 \text{ hvis } i\neq j
		\end{aligned}
	\right.
	\label{eq:vektor:prikkprodukt}
\end{displaymath}
der $i$ og $j$ kan være enten $x, y$ eller $z$. Kryssproduktet mellom enhetsvektorene er litt mer komplisert, men kan relativt enkelt finnes ved hjelp av høyrehåndsregelen. Siden alle enhetsvektorene har lengde 1 og står normalt på hverandre blir også kryssproduktet en vektor med lengde 1, bortsett fra hvis vi tar kryssprodukt av en enhetsvektor med seg selv---da får vi 0.
\begin{displaymath}
\begin{aligned}
	\ex\times\ex &= \ey\times\ey = \ez\times\ez = 0 \\
	\ex \times \ey & = \ez \\
	\ex \times \ez & = -\ey \\
	\ey \times \ez &= \ex 
	\label{eq:vektor:kryssprodukt}
\end{aligned}
\end{displaymath}
Hvis enhetsvektorene kommer i annen rekkefølge bruker vi den generelle regelen for kryssprodukt som gjelder for alle vektorer, inkludert enhetsvektorer, 
\begin{displaymath}
	\vec{a}\times\vec{b} = -\vec{b}\times\vec{a}
\end{displaymath}

\subsubsection{Regneeksempel}
Vi har hastighetsvektoren
\begin{displaymath}
	\vec{v} = (1~\mathrm{m/s})\ex - (3~\mathrm{m/s})\ey + (2~\mathrm{m/s})\ez
\end{displaymath}
og magnetfeltvektoren 
\begin{displaymath}
	\vec{B} = (1.2~\mathrm{T})\ex + (0.5~\mathrm{T})\ey - (0.7~\mathrm{T})\ez.
\end{displaymath}
Vi kan nå bruke relasjonene (\ref{eq:vektor:prikkprodukt}) og (\ref{eq:vektor:kryssprodukt}) til å regne ut prikk- og kryss-produktet av $\vec{v}$ og $\vec{B}$\footnote{$\vec{v}\times\vec{B}$ er et produkt som det ofte er behov for å beregne når man jobber med magnetiske krefter. $\vec{v}\cdot\vec{B}$ er ikke et produkt som vanligvis opptrer i fysiske beregninger, men det er ingenting som hindrer oss i å beregne dette produktet.}. Merk at beregningene her gjøres litt ekstra omstendelig---vanligvis vil man ikke skrive opp de leddene som gir 0 i det hele tatt, men for å vise mest mulig detaljer er alle ledd med her.
\begin{displaymath}
\begin{aligned}
	\vec{v}\cdot\vec{B} &= \left[ (1~\mathrm{m/s})\ex - (3~\mathrm{m/s})\ey + (2~\mathrm{m/s})\ez \right] \cdot
	\left[(1.2~\mathrm{T})\ex + (0.5~\mathrm{T})\ey - (0.7~\mathrm{T})\ez\right] \\
	&= (1.2~\mathrm{Tm/s})\ex\cdot\ex + (0.5~\mathrm{Tm/s})\ex\cdot\ey -  (0.7~\mathrm{Tm/s})\ex\cdot\ez \\
	&\quad- (3.6~\mathrm{Tm/s})\ey\cdot\ex- (1.5~\mathrm{Tm/s})\ey\cdot\ey+ (2.1~\mathrm{Tm/s})\ey\cdot\ez \\
	&\quad+ (2.4~\mathrm{Tm/s})\ez\cdot\ex+ (1.0~\mathrm{Tm/s})\ez\cdot\ey -  (1.4~\mathrm{Tm/s})\ez\cdot\ez \\
	&= 1.2~\mathrm{Tm/s} - 1.5~\mathrm{Tm/s} - 1.4~\mathrm{Tm/s} \\
	&= -1.7~\mathrm{Tm/s}.
\end{aligned}
\end{displaymath}
\begin{displaymath}
\begin{aligned}
	\vec{v}\times\vec{B} &= \left[ (1~\mathrm{m/s})\ex - (3~\mathrm{m/s})\ey + (2~\mathrm{m/s})\ez \right] \times
	\left[(1.2~\mathrm{T})\ex + (0.5~\mathrm{T})\ey - (0.7~\mathrm{T})\ez\right] \\
	&= (1.2~\mathrm{Tm/s})\ex\times\ex + (0.5~\mathrm{Tm/s})\ex\times\ey -  (0.7~\mathrm{Tm/s})\ex\times\ez \\
	&\quad- (3.6~\mathrm{Tm/s})\ey\times\ex- (1.5~\mathrm{Tm/s})\ey\times\ey+ (2.1~\mathrm{Tm/s})\ey\times\ez \\
	&\quad+ (2.4~\mathrm{Tm/s})\ez\times\ex+ (1.0~\mathrm{Tm/s})\ez\times\ey -  (1.4~\mathrm{Tm/s})\ez\times\ez \\
	&=(0.5~\mathrm{Tm/s}+3.6~\mathrm{Tm/s})\ex\times\ey + (-0.7~\mathrm{Tm/s}-2.4~\mathrm{Tm/s})\ex\times\ez \\
	&\quad+(2.1~\mathrm{Tm/s}-1.0~\mathrm{Tm/s})\ey\times\ez \\
	&=(4.1~\mathrm{Tm/s})\ez - (3.1~\mathrm{Tm/s})(-\ey) +(1.1~\mathrm{Tm/s})\ex \\
	&=(1.1~\mathrm{Tm/s})\ex + (3.1~\mathrm{Tm/s})\ey + (4.1~\mathrm{Tm/s})\ez
\end{aligned}
\end{displaymath}

\subsection{Enhetsvektorer i sylinderkoordinater}
Enhver bevegelse \emph{kan} beskrives ved hjelp av kartesiske koordinater, men noen bevegelse vil da se unødvendig komplisert ut. Dette gjelder blant annet sirkelbevegelser og skurebevegelser. Disse bevegelsene beskrivers bedre i sylinderkoordinater. Vi har da
\begin{itemize}
\item
 en koordinat som beskriver avstanden fra sentrum av bevegelsen. Denne koordinaten kalles ofte $r$.
\item
 en koordinat som beskriver vinkelen relativt til en (tilfeldig valgt) retning i planet der sirkelbevegelsen skjer. Denne koordinaten kalles ofte $\theta$.
\item
en koordinat som beskriver høyden over det planet vi tok utgangspunkt i. Denne koordinaten kalles ofte $z$. 
\end{itemize}
Vi definerer nå tre enhetsvektorer $\er, \eth$ og $\ez$. Enhetsvektoren $\ez$ er akkurat slik som enhetsvektorene diskutert i forbindelse med de kartesiske koordinatene, men de to andre er litt annerledes: $\er$ og $\eth$ peker nemlig i ulik retning avhengig av hvilket punkt vi ser på. Enhetsvektoren $\er$ peker i samme retning som forbindelseslinjen mellom origo og punktet vi ser på, mens $\eth$ står $90^\circ$ på $\er$ og $\eth$ med retningen bestemt slik at $\er, \eth$ og $\ez$ utgjør et høyrehendt system når de er tatt i den rekkefølgen. Dette betyr at når $z$-aksen peker oppover øker vinkelen \emph{mot} klokken. Vi kan nå uttrykke en vilkårlig vektor ved hjelp av enhetsvektorene vi har definert,
\begin{displaymath}
	\vec{a} = a_r\ex + a_\theta\eth + a_z\ez.
\end{displaymath}

Siden retningen til $\er$ og $\eth$ varierer fra punkt til punkt vil de påvirkes av derivasjon. 