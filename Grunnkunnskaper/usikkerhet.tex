\chapter{Usikkerhet og signifikante sifre}
I motsetning til i matematikk der vi som regel kan anta at vi har eksakt kjennskap til størrelsene vi jobber med har vi i fysikk generelt bare kjennskap til størrelsen med en viss presisjon. Dette skyldes at fysiske størrelser har opphav i målinger og der kan vi aldri ha uendelig presisjon. Skal vi gjøre ting helt korrekt skal størrelsen oppgis sammen med dens estimerte usikkerhet.\footnote{Heller ikke usikkerheten er eksakt bestemt, men det er som regel mulig å estimere denne på en systematisk måte.
} For eksempel kan en lengde oppgis som $x=(1.03\pm0.02)~\mathrm{m}$. Dette betyr det beste estimatet vi har på lengden er $x = 1.03~\mathrm{m}$. Usikkerheten $\pm0.02~\mathrm{m}$ betyr at vi mener å vite at hvis denne lengden blir målt mange ganger vil flesteparten\footnote{Det vanligste er å bruke en definisjon av usikkerheten som gjør at 67\% at målingene skal havne innenfor intervallet definert av usikkerheten.} av målingene skal vise mellom $x=1.01~\mathrm{m}$ og $x=1.05~\mathrm{m}$. 

I regneoppgaver forholder vi oss vanligvis litt mer avslappet til usikkerheten i størrelsene, men vi glemmer den ikke helt. Hvis f.eks. en energi oppgis som $2.52~\mathrm{J}$ tolkes det som at usikkerheten er i det siste sifferet. Det kan altså tenkes at den riktige verdien er $2.51~\mathrm{J}$ eller $2.55~\mathrm{J}$, men sannsynligvis ikke at avviket er så stort at den riktige verdien er $2.60~\mathrm{J}$. En tommelfingerregel for å få en fornuftig presisjon i svaret av en beregning er å ikke ta med flere sifre i sluttsvaret enn det var i den inn-verdien med færrest sifre.

Noen få ord om hvordan man teller sifre: Hvis vi teller fra venstre er det første sifferet som er ulik 0 det første vi teller. Deretter teller alle sifre med i regnskapet. Eksempler:
\begin{itemize}
	\item 1.54 -- tre sifre
	\item 0.03 -- ett siffer
	\item$1.3\times10^6$ -- to sifre
\end{itemize}
Merk at hvis du får svaret $238~\mathrm{m}$ og kun skal ha med to sifre må svaret skrives som $2.4\time10^2~\mathrm{m}$ for å følge denne regelen siden å skrive det avrundete tallet $240~\mathrm{m}$ fremdeles er å skrive tre sifre.

\section{Mer om usikkerhet}
Noen ganger er ikke tommelfingerregelen om antall signifikante sifre tilstrekkelig. For eksempel må usikkerhet behandles skikkelig hvis man skriver en vitenskaplig artikkel eller en teknisk rapport. Også i labøvinger er det vanlig å kreve at usikkerheten behandles på en ordentlig måte. 

Vi skal her se kort på hvordan man propagerer usikkerhet fra inntallene til svaret. Å gjøre dette helt generelt er komplisert, så vi vil her gjøre to antakelser som forenkler problemet:
\begin{itemize}
\item
Usikkerheten til alle variablene er ukorrelerte. Det vil si at selv om usikkerheten på en variabel, så vil ikke de andre 	usikkerhetene påvirkes.
\item
Alle usikkerhetene kommer fra en gaussisk fordeling. 
\end{itemize}
Anta at vi har måledata $a\pm\sigma_a,b\pm\sigma_b,c\pm\sigma_c,d\pm\sigma_d$ som kombineres til et resultat $f(a,b,c,d)$. Vi ønsker nå å finne usikkerheten, $\sigma_f$, til dette resultatet. Det finner vi via formelen:
\begin{displaymath}
	\sigma_f^2 =
	\left(\frac{\partial f}{\partial a}\right)^2\sigma_a^2 + \left(\frac{\partial f}{\partial b}\right)^2\sigma_b^2 + 
	\left(\frac{\partial f}{\partial c}\right)^2\sigma_c^2 + \left(\frac{\partial f}{\partial d}\right)^2\sigma_d^2 
\end{displaymath}
For å gjøre det mer konkret skal vi se på noen spesialtilfeller, og for enkelhets skyld nøyer vi oss med to inn-variabler, men resultatene kan enkelt utvides til flere variabler.
\subsubsection*{Addisjon}
$f(a,b) = a + b,\quad	\sigma_f^2 = \sigma_a^2 + \sigma_b^2$
\subsubsection*{Subtraksjon}
$f(a,b) = a - b, \quad	\sigma_f^2 = \sigma_a^2 + \sigma_b^2$ \\
Merk at siden usikkerhetene adderes i kvadratur her, mens $f$ kommer som resultat av en subtraksjon kan den relative usikkerheten på resultatet, $\sigma_f/f$, bli veldig mye større enn de relative usikkerhetene på inndataene. Det er derfor en fordel, om mulig, å organisere målingene og beregningene sine slik at subtraksjon unngås.
\subsubsection*{Multiplikasjon}
$f(a,b) = a\cdot b, 
\quad  \left(\frac{\sigma_f}{f}\right)^2 = \left(\frac{\sigma_a}{a}\right)^2 + \left(\frac{\sigma_b}{b}\right)^2$
\subsubsection*{Multiplikasjon med en tall uten usikkerhet}
$f(a) = 2a, \quad \sigma_f = 2\sigma_a$
\subsubsection*{Divisjon}
$f(a,b) = \frac{a}{b}, \quad 
\quad  \left(\frac{\sigma_f}{f}\right)^2 = \left(\frac{\sigma_a}{a}\right)^2 + \left(\frac{\sigma_b}{b}\right)^2$\\
På samme måte som subtraksjon vil også divisjon kunne blåse opp den relative usikkerheten.