\chapter{Radiokativitet}
De fleste isotoper er stabile---det vil si de endrer ikke karakter med mindre de blir utsatt for en ytre påvirkning som endrer dem. Imidlertid er det noen isotoper som er ustabile. Slike isotoper vil spontant endre seg enten ved å
\begin{itemize}
\item
sende ut en heliumkjerne bestående av 2 protoner og to elektroner ($\alpha$-partikkel),
\item
omgjøre et nøytron til et proton samtidig som det sendes ut et elektron ($\beta$-partikkel),
\item
omgjøre et proton til et nøytron samtidig som det sendes ut et positron ($\beta+$-partikkel),
\item
omgjøre et proton til et elektron ved å fange et av elektronene rundt isotopkjernen ($\beta$-innfanging),
\item
spaltes i to isotopkjerner + et antall nøytroner (fisjon).
\end{itemize}
Enkelte av prosessene etterlater isotopkjernen i en eksitert tilstand. Da vil den etter kort tid henfalle til grunntilstanden samtidig som det sendes ut et høy-energetisk foton ($\gamma$-stråling). Hvilken prosess som skjer og hvor lenge isotopet eksisterer før det skjer avhenger av hvilket isotop vi starter med.

\section{Levetiden til ustabile isotoper}
Henfallet av ustabile isotoper er såvidt vi vet en fullstendig tilfeldig prosess---det vil si at det er ikke på noen måte mulig å forutsi akkurat når henfallet vil skje. Hver enkelt type ustabilt isotop har imidlertid en gjennomsnittlig levetid, $\tau$.  Hvis vi starter med $N_0$ atomer\footnote{Vi antar i denne diskusjonen at antall atomer vi starter med er stort nok til at statistiske fluktuasjoner er neglisjerbare.} vil antallet atomer som ikke har henfalt etter tid $t$ være
\begin{equation}
	N(t) = N_0e^{-t/\tau}.
	\label{eq:radioaktiv:henfall}
\end{equation}
Ofte snakker man om halvveringstiden til de ustabile isotopene i stedet for levetiden. Halvveringstiden $t_{1/2}$ er den tiden det tar før halvparten av isotopene har henfalt. Fra ligning (\ref{eq:radioaktiv:henfall}) finner vi da
\begin{displaymath}
	\half N_0 = N_0e^{-t_{1/2}/\tau}
\end{displaymath}
som viser at sammenhengen mellom levetid og halvveringstid er
\begin{equation}
	t_{1/2} = \tau\ln 2.
\end{equation}
For ulike ustabile isotoper kan halvveringstiden til variere fra små brøkdeler av et sekund til lenger enn universets levetid. 

\section{Kilder til radioaktiv stråling}


\section{$\alpha$-stråling}
$\alpha$-stråling består av heliumkjerner---det vil si atomkjerner med 2 protoner og 2 nøytroner---og de har da elektrisk ladning $+2e$. Den elektriske ladningen gjør at $\alpha$-partiklene kan ionisere atomer. I tillegg vil de---siden de består av protoner og nøytroner---også vekselvirke via den sterke kjernekraften. Kombinasjonen av stor masse (sammenlignet med massen f.eks.~elektroner) og sterk kjernekraft gjør at det kreves lite materiale for å stanse $\alpha$-stråling. {\color{red}[Litt kvantitativt om hva lite materiale betyr]} Den praktiske konsekvensen av dette er at det er enkelt å skjerme seg mot $\alpha$-stråling. Til gjengjeld vil $\alpha$-stråling avsette energien sin i et konsentrert område om den ikke er skjermet, og kan da gi betydelig vev-skade. Dette gjør at radioaktive stoffer som henfaller ved å sende ut $\alpha$-stråling først og fremst er farlig hvis de kommer direkte kontakt med huden eller kommer inn i kroppen. Sistnevnte er spesielt aktuelt med radon-gass som pustes inn.

\section{$\beta$-stråling}