\documentclass[a4paper,norsk,12pt]{book}

\usepackage[norsk]{babel}
\usepackage{enumitem}
\usepackage{color}
\usepackage{amsmath}
\usepackage{wrapfig}
\usepackage{graphicx}
\usepackage[utf8]{inputenc}
\def\d{\ensuremath{\text{d}}}
\def\half{\ensuremath{\frac{1}{2}}}

\begin{document}
\tableofcontents

\chapter{Introduksjon}
Stråling er et begrep som favner ganske vidt, men vanligvis når man snakker om stråling er det snakk om enten elektromagnetisk stråling eller partikkelstråling. Unntaksvis snakkes det også om akkustisk stråling (lyd). Felles for alle typer stråling er at det er overføring av energi med stor hastighet og i rette linjer. For partikkelstråling overføres også masse.

\section{Elektromagnetisk stråling}
Elektromagnetisk stråling er energi som overføres i form av elektromagnetiske bølger. Avhengig av bølgelengde karakteriserers den elektromagnetiske strålingen som (fra lengst til kortest bølgelengde)
\begin{itemize}
	\item Radiobølger
	\item Mikrobølger
	\item Infrarød (IR) stråling
	\item Lys
	\item Ultrafiolett (UV) stråling
	\item Røntgenstråling
	\item Gamma-stråling
\end{itemize}
Bølgelengdene til elektromagnetisk stråling følger en kontinuerlig skala. Dette innebærer at det ikke er skarpt definerte skiller mellom hvilke bølgelengder som faller inn i hvilken av de ulike kategoriene. På grunn av måten gamma-stråling vanligvis dannes tas denne type stråling ofte med under radioaktiv stråling selv om den ikke er vesensforskjellig fra andre typer elektromagnetisk stråling.

Alle typer elektromagnetiske bølger beveger seg alltid med hastighet $c = 3.00\times10^8~\mathrm{m/s}$ når forplantningen skjer i vakuum. Hastigheten $c$ omtales ofte som lyshastigheten (underforstått lyshastigheten i vakuum). Når elektromagnetiske bølger forplanter seg gjennom et medium vil de bevege seg med hastighet mindre enn $c$. Hvor stor hastighetsreduksjonen  er avhengig av hvilket medium de elektromagnetiske bølgene beveger seg i og i en del tilfeller også av bølgelengden.

\section{Partikkelstråling}
Partikkelstråling er atomkjerner eller sub-atomære partikler som beveger seg med stor hastighet. De viktigste naturlige kildene til partikkelstråling er radioaktive prosesser og kosmisk stråling. Kunstig akselererte partikler brukes i forskningsøyemed og til kreftbehandling.

\section{Biologisk effekt av stråling}
Alle typer stråling vil avsette energi i biologisk materiale, men hvordan og i hvilken grad energien avsettes har stor betydning for den biologiske effekten. I dette kompendiet diskuteres i hovedsak negative biologiske effekter av stråling, og da er det først og fremst høy-energetisk stråling som er av interesse. Hvis strålingen---uavhengig av type--- har tilstrekkelig energi til å slå løs elektroner fra atomkjernen de er bundet til kaller vi strålingen ioniserende. I det elektromagnetiske spekteret er det røntgen-stråling og gamma-stråling som er ioniserende, mens alle typer partikkelstråling der partiklene er ladde kan være ioniserende gitt at energien er tilstrekkelig stor. Siden det er elektronene som binder atomene sammen til molekyler vil ioniserende stråling kunne forårsake skade på biologisk materiale. 

DNA-molekyler avhenger av hydrogenbindinger som er så svake at det kreves langt mindre energi å bryte de opp enn det som skal til for å ionisere atomene. Dette gjør at også UV-stråling kan ha betydelig negativ effekt på biologisk materiale.

Elektromagnetisk stråling med kortere bølgelengde enn UV har ingen akkutt negativ virkning slik som den høyenergetiske strålingen. Men også denne strålingen vil avsette energi, og om denne energiavsetningen blir stor nok vil det gi oppvarming av vevet som potensielt sett kan være skadelig. 

\chapter{Radiokativitet}
De fleste isotoper er stabile---det vil si de endrer ikke karakter med mindre de blir utsatt for en ytre påvirkning som endrer dem. Imidlertid er det noen isotoper som er ustabile. Slike isotoper vil spontant endre seg enten ved å
\begin{itemize}
\item
sende ut en heliumkjerne bestående av 2 protoner og to elektroner ($\alpha$-partikkel),
\item
omgjøre et nøytron til et proton samtidig som det sendes ut et elektron ($\beta$-partikkel),
\item
omgjøre et proton til et nøytron samtidig som det sendes ut et positron ($\beta+$-partikkel),
\item
omgjøre et proton til et elektron ved å fange et av elektronene rundt isotopkjernen ($\beta$-innfanging),
\item
spaltes i to isotopkjerner + et antall nøytroner (fisjon).
\end{itemize}
Enkelte av prosessene etterlater isotopkjernen i en eksitert tilstand. Da vil den etter kort tid henfalle til grunntilstanden samtidig som det sendes ut et høy-energetisk foton ($\gamma$-stråling). Hvilken prosess som skjer og hvor lenge isotopet eksisterer før det skjer avhenger av hvilket isotop vi starter med.

\section{Levetiden til ustabile isotoper}
Henfallet av ustabile isotoper er såvidt vi vet en fullstendig tilfeldig prosess---det vil si at det er ikke på noen måte mulig å forutsi akkurat når henfallet vil skje. Hver enkelt type ustabilt isotop har imidlertid en gjennomsnittlig levetid, $\tau$.  Hvis vi starter med $N_0$ atomer\footnote{Vi antar i denne diskusjonen at antall atomer vi starter med er stort nok til at statistiske fluktuasjoner er neglisjerbare.} vil antallet atomer som ikke har henfalt etter tid $t$ være
\begin{equation}
	N(t) = N_0e^{-t/\tau}.
	\label{eq:radioaktiv:henfall}
\end{equation}
Ofte snakker man om halvveringstiden til de ustabile isotopene i stedet for levetiden. Halvveringstiden $t_{1/2}$ er den tiden det tar før halvparten av isotopene har henfalt. Fra ligning (\ref{eq:radioaktiv:henfall}) finner vi da
\begin{displaymath}
	\half N_0 = N_0e^{-t_{1/2}/\tau}
\end{displaymath}
som viser at sammenhengen mellom levetid og halvveringstid er
\begin{equation}
	t_{1/2} = \tau\ln 2.
\end{equation}
For ulike ustabile isotoper kan halvveringstiden til variere fra små brøkdeler av et sekund til lenger enn universets levetid. 

\section{Kilder til radioaktiv stråling}


\section{$\alpha$-stråling}
$\alpha$-stråling består av heliumkjerner---det vil si atomkjerner med 2 protoner og 2 nøytroner---og de har da elektrisk ladning $+2e$. Den elektriske ladningen gjør at $\alpha$-partiklene kan ionisere atomer. I tillegg vil de---siden de består av protoner og nøytroner---også vekselvirke via den sterke kjernekraften. Kombinasjonen av stor masse (sammenlignet med massen f.eks.~elektroner) og sterk kjernekraft gjør at det kreves lite materiale for å stanse $\alpha$-stråling. {\color{red}[Litt kvantitativt om hva lite materiale betyr]} Den praktiske konsekvensen av dette er at det er enkelt å skjerme seg mot $\alpha$-stråling. Til gjengjeld vil $\alpha$-stråling avsette energien sin i et konsentrert område om den ikke er skjermet, og kan da gi betydelig vev-skade. Dette gjør at radioaktive stoffer som henfaller ved å sende ut $\alpha$-stråling først og fremst er farlig hvis de kommer direkte kontakt med huden eller kommer inn i kroppen. Sistnevnte er spesielt aktuelt med radon-gass som pustes inn.

\section{$\beta$-stråling}


\chapter{Ultrafiolett stråling}
Ultrafiolett stråling har for lite energi-rike fotoner til å ionisere atomer, men kan likevel gjøre skade på kroppsvev. Den viktigste skademekanismen er produksjon av tymin-dimerer i DNA-molekylene våre \cite{tve-uv}. Tymin-dimerer er sammenkobling av to tymin-baser som er plassert ved siden av hverandre i DNA-strengen. Produksjon av tymin-dimerer hindrer replisering av DNA-strengen. Grunnen til at UV-lys kan forårsake slike skader i DNA'et er at basene fra hver side av DNA-molekylet er bundet sammen med hydrogenbindinger som det kreves relativt lite energi for å bryte.

\end{document}