\chapter{Kvantemekanikk}

Tidlig i fysikkundervisningen lærer vi at ved å se på summen av krefter på som virker på et objekt kan vi finne ut hvordan det vil bevege seg, $\sum\vec{F} = m\vec{a}$. Hvis vi prøver å bruke denne ligningen til å forutse hvordan et elektron beveger seg vil vi ofte\footnote{Det finnes tilfeller der Newtons lover gir en god beskrivelse av bevegelsen til elektroner, men i det generelle tilfellet er det ikke slik.} få feil resultat. Det viser seg at bevegelsen til elektroner---og andre tilstrekkelig små partikler og systemer av partikler---må beskrives på en helt annen måte. Det er det kvantemekanikken dreier seg om. Denne teksten vil ikke forsøke å gi en fullstendig innføring i kvantemekanikk, men bare diskutere litt generelle aspekter ved kvantemekanikken og gi en litt grundigere diskusjon av de elementene som er nødvendig for å forstå hvordan kvantedatamaskiner virker.

\section{Dobbelspalteeksperimentet}
Vi begynner med et kjent eksperiment som svært tydelig viser forskjellen mellom kvantemekanikk og klassisk mekanikk: dobbeltspalteeksperimentet. Før vi introduserer kvantemekanikken, la oss se på en klassisk versjon av eksperimentet. Figur \ref{fig:kvante:klassiskspalte} viser en skisse over oppsettet. Helt til venstre har vi en liten "kanon" som skyter ut små kuler i tilfeldige retninger. Helt til høyre har vi en plate som er i stand til å føle hvor den blir truffet. Midt mellom kanonen og detektorplaten er det en skjerm med to huller som er relativt små, men store nok til at kulene kan passere gjennom. For å forenkle diskusjonen skal vi betrakte systemet som om det kun har de to dimensjonene som er vist på figuren. Å ta med den tredje dimensjonen som finnes i den virkelige verden forandrer verken prinsippet i diskusjonen eller det essensielle i konklusjonen, så å ta med den ville bare vært en unødvendig komplikasjon.

Siden kulene blir skutt ut i tilfeldige retninger vil de fleste ikke treffe noen av hullene. Vi skal derimot kun bry oss om de som faktisk treffer et av hullene. En kule som treffer et hull kan enten gå rett gjennom hullet og fortsette rett frem i den retningen det hadde, eller den kan treffe kanten av hullet å få endret retningen sin litt eller mye. I figur \ref{fig:kvante:klassisktreff} er det skissert hvordan treffpunktene på detektorplaten fordeler seg. Skissen lengst til venstre viser fordelingen av kun de kulene som gikk gjennom det øvre hullet\footnote{Vi bør egentlig tenke oss at systemet ligger flatt slik at ikke tyngdekraften påvirker fordelingen, men siden det på figuren er et øvre og et nedre hull er det enklest å omtale de på denne måten.}. Skissen i midten viser fordelingen av kun de kulene som gikk gjennom det nederste hullet. Skissen lengst til høyre viser fordelingen av alle treffpunktene. Den siste fordelingen er naturlig nok lik summen av de to første. Dette betyr at hvis vi ønsker å finne sannsynligheten for at et bestemt punkt på detektorplaten blir truffet kan vi enkelt regne det ut som
\begin{displaymath}
	p(x) = p_1(x) + p_2(x).
\end{displaymath}
Her viser koordinaten $x$ til hvilket punkt på detektorplaten vi vil undersøke. Vi kan for eksempel la $x=0$ være punktet som ligger rett bak midt mellom hullene, og la $x$ måle hvor høyt over eller lavt under dette punktet er. $p_1(x)$ svarer til sannsynligheten for at en kule går gjennom det øvre hullet og så treffer punktet $x$. Tilsvarende er $p_2(x)$ sannsynligheten for at en kule går gjennom det nedre hullet og treffer punktet $x$. 
\begin{figure}[tp]
	\includegraphics[width=\textwidth]{./dobbeltspalte1}
	\caption{Skisse av dobbeltspalte-eksperimentet utført med klassiske kuler. De grå linjene indikerer eksempler på baner kulene tar fra kilden til detektorplaten. Kurven som er skissert på skjermen viser fordelingen av treffpunkter. Skissen til venstre og i midten viser fordelingen av treff dersom et av hullene er blokkert. Skissen til høyre viser fordelingen av treff dersom begge huller er åpne. Merk at treffordelingen med begge huller åpne ganske enkelt er summen av treffordelingene der et av hullene er blokkert.}
	\label{fig:kvante:klassisktreff}
\end{figure}

Vi bytter nå ut kanonen vår med en "elektronkanon"---en kilde som skyter ut elektroner i tilfeldige retninger. Oppsettet for øvrig er likt---selvfølgelig med den tilpasning at vi nå velger en skjerm med huller som er mye mindre siden elektronene er mye mindre enn kulene, og en annen type detektorplate siden det er elektroner som skal detekteres nå. Fremdeles skal hullene være store nok til at elektronene kan gå gjennom, men små nok til at det er en vesentlig sannsynligehet for at de vil treffe kanten av hullet og endre retning. Naivt sett skulle man forvente at vi får se akkurat det samme som i tilfellet med kuler, men det er absolutt ikke det som skjer.

Med kun ett hull åpent skjer ikke den store endringen, men å ha begge hullene åpne endrer alt. Dette er illustrert i figur \ref{fig:kvante:kvantetreff}. Hvis vi insisterer på å tolke elektronene som kuler som beveger seg fra kilden, gjennom et av hullene og til detektorplaten gir ikke denne treffordelingen noen mening. Hvis vi dermed tolker elektronene som bølger er fordelingen akkurat slik man skulle forventet, se figur \ref{fig:kvante:interferens}. Bølgen brer seg utover fra kilden, og de delene av bølgefronten som treffer et hull fortsetter på andre siden av skjermen. Hullene blir da essensielt sett nye kilder til bølger som brer seg utover. Når bølgene fra de to hullene treffer hverandre vil resultatet bli en sum av de to bølgene: Hvis to bølgetopper treffer hverandre blir resultatet dobbelt så stort som den enkelte bølgen i seg selv. Hvis en bølgetopp treffer en bølgebunn blir det ikke noe utslag i det hele tatt. Dette forklarer hvorfor treffordelingen på detektorskjermen viser en stor topp i midten, og også noen områder uten treff i det hele tatt.

\begin{figure}[tp]
	\includegraphics[width=\textwidth]{./dobbeltspalte2}
	\caption{Skisse av dobbeltspalte-eksperimentet utført med elektroner. De grå linjene indikerer eksempler på baner elektronene tar fra kilden til detektorskjermen. Kurven som er skissert på skjermen viser fordelingen av treffpunkter. Merk at fordelingen når et av hullene er blokkert ser likt ut som i det klassiske tilfellet, men når begge hullene er åpne blir fordelingen helt annerledes.}
	\label{fig:kvante:kvantetreff}
\end{figure}

Denne bølgetolkningen av elektronet er imidlertid ikke hele historien. Hvis vi nå gjør eksperimentet og sørger for at kun ett elektron passerer gjennom apparatet til enhver tid får vi fremdeles treffordelingen som er indikert i figur \ref{fig:kvante:kvantetreff}. Men hvert enkelt elektron treffer ikke litt her og litt der på detektorskjermen slik en bølge ville gjort. Hvert elektron treffer detektorskjermen på et veldefinert sted. Forklaringen på hvorfor treffordelingen blir som den blir må altså inneholde både denne bølgelike bevegelsen gjennom systemet, og den partikkel-like måten å treffe detektorskjermen. 

En løsning på denne utfordringen er å beskrive elektronet med en ``bølge\-funksjon'' som vi skal kalle $\psi$. Denne bølgefunksjonen beskriver hvordan elektronet beveger seg fra kilden til detektorskjermen, og i likhet med bølgen i figur \ref{fig:kvante:interferens}---og også i likhet med bølger på vann---er den ikke lokalisert til ett sted i rommet om gangen slik en partikkel er. Anvendt på elektronet som går gjennom de to hullene får vi da en situasjon som er temmelig lik det vi ser med en vanlig bølge: Hvert av hullene blir effektivt sett kilden til hver sin bølgefunksjon som vi kan kalle $\psi_1$ og $\psi_2$. I et punkt på skjermen der $\psi_1$ og $\psi_2$ er i fase forsterker de hverandre, mens i punkter der de er i motfase utslukker de hverandre. Sannsynligheten for å måle at elektronet treffer i akkurat denne posisjonen kan da beregnes som\footnote{At amplituden er kvadrert stemmer overens med klassiske bølger: Intensiteten, altså energi overført per tid, til en bølge avhenger ikke av amplituden, men kvadratet av amplituden.}\footnote{Når vi her har skrevet $|\psi_1 + \psi_2|^2$ og ikke $(\psi_1+\psi_2)^2$ er det fordi vi i noen tilfeller må regne med en kompleks bølgefunksjon. Sannsynligheten vi beregner må være reell, så dermed må vi ta kvadratet av absoluttverdien i stedet for kvadratet av amplituden selv.}.
\begin{displaymath}
	P = |\psi_1+\psi_2|^2.
\end{displaymath}
Hvis du leste den siste setningen nøye ser du et stort avvik fra klassiske bølger: En klassisk bølge ville truffet ulike deler av skjermen med ulik intensitet. Elektronet, som er beskrevet av bølgefunksjonen $\psi$ treffer imidlertid skjermen på et veldefinert sted. Matematisk sett er ikke dette et problem med kvantemekanikken, men det faktum at elektronet beveger seg som en bølge, men treffer som en partikkel skaper problemer når vi forsøker å \emph{forstå} kvantemekanikken. Dette er et interessant problem å diskutere, men jeg vi likevel ikke bruke plass på å ta den diskusjonen her. 

Litt mer detaljer om bølgefunksjonen $\psi$ blir diskutert i avsnitt \ref{sec:kvante:psi} og \ref{sec:kvante:schrodinger}.

\subsection{Måling av hvilket hull elektronet går gjennom}
Beskrivelsen over forteller oss at elektronet går gjennom begge hullene på vei fra kilden til detektorskjermen. Men kan et elektron virkelig få to veier samtidig? Og hva hvis vi setter opp en eller annen slags detektor som måler hvilket hull det passerer gjennom, vil detektoren da fortelle oss at det går gjennom begge hullene? En måte å måle hvilket hull elektronet passerer gjennom uten å stoppe elektronet er å sende en lysstråle forbi hullene slik som vist på figur \ref{fig:kvante:hulldetektor}. Når lyset treffer elektronet vil noe av det spres bort fra stråleretningen. Det vil si at ved å se hvor det spredte lyset kommer fra kan vi finne ut hvilket hull elektronet kom ut av. Så hva viser målingen? Kommer elektronet ut av begge hullene samtidig (husk at vi lar kun ett elektron passere gjennom systemet om gangen)? Nei. Målingen viser at elektronet passerer gjennom enten det ene hullet eller det andre, og ikke begge samtidig. Men nå har det skjedd noe annet også: interferensmønsteret er borte og treffordelingen er lik den klassiske som var vist i figur \ref{fig:kvante:klassisktreff}. For å oppsummere: Dersom vi måler hvilket hull elektronet går gjennom oppfører det seg "vanlig", altså slik som vi er vant med fra klassisk fysikk. Dersom vi ikke måler hvilket hull elektronet går gjennom ser det ut til å gå gjennom begge hullene samtidig og interfererer med seg selv.

\subsection{Flere skjermer og flere hull}
Det er ingen grunn til at eksperimentet skal utføres med bare \'en skjerm med nøyaktig to hull. Vi kan for eksempel legge til en skjerm til som også har to hull. Da har elektronet fire muligheter: $(A_1, B_1)$, $(A_1, B_2)$, $(A_2,B_1)$ og $(A_2,B_2)$, der bokstaven ($A$ eller $B$) viser til første eller andre plate og indeksen ($1$ eller $2$) viser til øverste eller nederste hull på platen. For å finne sannsynligheten for at elektronet skal treffe i et bestemt punkt på detektorskjermen må vi summere over bølgefunksjonen $\psi$ som hører til hver av de fire alternativene, og til slutt regne ut kvadratet av absoluttverdien:
\begin{displaymath}
	P(x) = |\psi(x)|^2 = |\psi_{11}(x) + \psi_{12}(x) + \psi_{21}(x) + \psi_{22}(x)|^2.
\end{displaymath}
Men det er heller ingen grunn til å stoppe her. For det første kan vi bore flere hull i hver skjerm. For det andre kan vi legge til flere skjermer. Hvis vi fortsetter å bore huller og legge til skjermer har vi til slutt uendelig mange skjermer med uendelig mange hull. Men det er jo det samme som å ikke ha noe i mellom. Denne beskrivelsen antyder at for å beregne sannsynligheten for at et elektron skal treffe på et bestemt sted må vi summere---eller i realiteten integrere siden det nå er et kontinuerlig sett av muligheter---over alle tenkbare baner fra kilde til detektorskjerm. Uttrykket vi ender opp med da ser ganske annerledes ut, men representerer egnetlig bare denne summen over uendelig mange muligheter:
\begin{displaymath}
	P(a,b) = \int_a^be^{(i/\hbar)S[b,a]}{\cal D}r(t)
\end{displaymath}
Her er $P(a,b)$ sannsynligheten for å starte i punkt $a$ og ende opp i punkt $b$. $S[a,b]$ er en funksjon som kalles \emph{virkningen} og som koder den informasjonen som ligger i $\psi$---altså bidraget til sannsynligheten for de ulike veiene $r(t)$ som tar oss fra $a$ til $b$. $\hbar$ er Planck's konstant og $i=\sqrt{-1}$. 
Vi skal ikke se på hvordan man beregner dette integralet her, men kun diskutere det litt videre kvalitativt. 

Det første som er viktig å vite er at selv om alle mulige veier fra start til mål er med i integralet bidrar ikke alle veier like mye. Generelt vil det være slik at den veien som en klassisk partikkel ville fulgt er den som gir det største bidraget til integralet, mens jo lengre bort fra denne veien en kommer jo mindre blir bidraget. Ut fra dette kan vi faktisk forstå hvorfor store objekter ikke ser ut til å følge kvantemekanikkens lover, men oppfører seg i stedet ``vanlig''. Et makroskopisk objekt er bygget opp av en masse små protoner, nøytroner og elektroner som alle følger kvantemekanikkens lover. Når vi regner ut sannsynlighetsbidraget som kommer fra at det makroskopiske objektet tar en vei som ligger bort fra den klassiske er det to tilfeller vi må se på:
\begin{enumerate}
\item
Alle de små partiklene som bygger opp det makroskopiske objektet gjør samtidig et avvik fra den klassiske veien. For å finne sannsynligheten for at dette skjer må vi multiplisere sammen sannsynligheten for at hver enkelt av dem tar denne ``omveien''. Siden antallet atomer som bygger opp makroskopiske objekter er så stort trenger ikke sannsynligheten for hver enkelt å være spesielt liten før produktet av alle sannsynlighetene blir forsvinnende lite.
\item
\'En eller noen av de små partiklene som bygger opp det makroskopiske objektet har et avvik fra den klassiske veien samtidig som resten følger den klassiske veien. Da må vi ta med sannsynligheten for at noen av partiklene skal bevege seg langt bort fra de andre, og igjen ender vi raskt opp med en forsvinnende liten sannsynlighet.
\end{enumerate}
Konklusjonen blir uansett at for makroskopiske objekter vil vi aldri kunne observere avvik fra den klassiske banen, selv om de i realiteten styres av kvantemekanikkens lover.

Det andre som er viktig å merke er at integralet innholder den samme muligheten for interferens som summene over komponenter av $\psi$ som vi så på tidligere. Siden muligheten for intereferens er det som utgjør den store forskjellen mellom klassisk mekanikk og kvantemekanikk er dette noe vi må se mer på.

\section{Interfereres eller ikke?}
La oss for enkelhets skyld vende tilbake til eksperimenter der det finnes kun to mulige veier fra start til mål. Dette kan være dobbeltspalteeksperimentet slik det først ble beskrevet, eller andre mulige eksperimenter som vi snart skal se eksempler på. Hvis vi ikke har interferens er sannsynligheten for å gå fra $a$ til $b$ 
\begin{displaymath}
	P(a,b) = P_1(a,b) + P_2(a,b)
\end{displaymath}
der $P_1(a,b)$ og $P_2(a,b)$ er de individuelle sannsynlighetene for de to ulike veiene. Selv om vi ikke har interferens kan vi fremdeles uttrykke dette med en bølgefunksjon $\psi$:
\begin{displaymath}
	P(a,b) = |\psi(a,b)|^2 = |\psi_1(a,b)+\psi_2(a,b)|^2 = |\psi_1(a,b)|^2 + |\psi_2(a,b)|^2
\end{displaymath}
der det essensielle er at kryssleddene fra å kvadrere absoluttverdien av summen av $\psi$ for de ulike veiene er lik 0. Det andre tilfellet er når vi har interferens og kryssleddene ikke forsvinner: 
\begin{displaymath}
	P(a,b) = |\psi(a,b)|^2 = |\psi_1(a,b)+\psi_2(a,b)|^2 \neq |\psi_1(a,b)|^2 + |\psi_2(a,b)|^2
\end{displaymath}
som betyr at 
\begin{displaymath}
	P(a,b) \neq P_1(a,b) + P_2(a,b).
\end{displaymath}

Hva er det som avgjør om vi ender opp med interferens eller ikke? I dobbelspalteeksperimentet med elektroner fikk vi interferens da vi ikke sjekket hvilket hull elektronet gikk gjennom, men interferensen forsvant straks vi målte hvor elektronet gikk. Ville resultatet blitt det samme om vi målte hvor elektronet gikk, men ikke så på resultatet av målingene?\footnote{Dette spørsmålet kan virke tåpelig, men kvantemekanikken er så annerledes enn det vi er vant med at denne type spørsmål har vært diskutert grundig blant både fysikere og filosofer.} Det korte svaret er at det blir ingen interferens dersom eksperimentet er slik at hvilket alternativ vi ender opp med kunne ha blitt bestemt. Hvis det ikke kunne ha blitt bestemt får vi interferens. Måling av hvilket hull elektronet passerer gjennom er et eksempel på noe som gjøre at vi ``kunne ha bestemt'' hvilket alternativ vi endte opp med, men også en tilstrekkelig sterk vekselvirkning med et makroskopisk objekt uten at det er tenkt som noen måling vil ha samme effekt.

\begin{figure}[tp]
\begin{center}
	\includegraphics{./spredningseksperiment}
\end{center}
\caption{Skisse av et ekseriment der, like eller ulike, partikler blir skutt ut fra kildene $A$ og $B$ med lik bevegelsesmengde. Siden bevegelsesmengden er lik vil de etter å ha kollidert fly ut i motsatt retning av hverandre. Hvis denne retningen er slik at de treffer detektorene 1 og 2 registrerer vi treffet.}
\label{fig:kvante:spredning}
\end{figure}

Det er også et annet instruktivt eksempel på hva som skiller situasjoner med og uten interfererende alternativer, men da må vi først innføre et nytt eksperiment skissert i figur \ref{fig:kvante:spredning}. La $A$ og $B$ være to kilder til partikler (f.eks. elektroner eller protoner) og 1 og 2 være detektorer som måler om en partikkel ender opp der. Hvis vi sørger for at partiklene blir skutt ut med lik bevegelsesmengde vil de etter å ha kollidert med hverandre fly ut fra kollisjons stedet i stikk motsatt retning av hverandre. Retningen trenger derimot ikke å være tilbake til der de kom fra---den kan derimot være hvilken retning som helst, og den vil heller ikke være lik fra forsøk til forsøk. Detektorene 1 og 2 registrerer når partiklene kommer den veien. Ved å gjøre forsøket mange ganger kan vi altså måle sannsynligheten for at retningen etter kollisjonen er slik at detektorene blir truffet.

I tråd med det vi har lært om kvantemekanikken starter vi ikke med å beregne sannsynligheten direkte, men med å beregne amplituden---altså verdien av $\psi$---som svarer til et utfall. Vi skal ikke se på hvordan vi beregner denne amplitudene, men kun på hvordan vi kombinerer amplitudene til å finne sannsynligheten. Anta derfor at vi kjenner
\begin{align*}
	\psi(A1, B2) &\text{ amplituden for at partikkel fra A ender opp i detektor 1,} \\
	&\text{ og partikkel fra B ender opp i detektor 2}. \\
	\psi(A2, B1) &\text{ amplituden for at partikkel fra A ender opp i detektor 2,} \\
	&\text{ og partikkel fra B ender opp i detektor 1}. \\
\end{align*}
Anta først at $A$ og $B$ skyter ut ulike partikler, for eksempel elektroner fra $A$ og protoner fra $B$. Da er det en målbar forskjell om det er elektronet eller protonet som havner i detektor 1. Sannsynligheten for å få en partikkel i hver av detektorene regnes derfor ut uten noen interferens,
\begin{displaymath}
	P = |\psi(A1, B2)|^2 + |\psi(A2, B1)|^2
\end{displaymath}
uansett om vi faktisk velger å undersøke hva slags partikkel som ender opp i hvilken detektor eller ikke. Dette endrer seg om $A$ og $B$ skyter ut like partikler, f.eks., en heliumkjerne fra hver kilde. Dersom en heliumkjerne treffer detektor 1 kan vi ikke avgjøre om den kom fra kilde $A$ eller $B$, og vi må derfor ta hensyn til interferens når vi beregner sannsynligheten:
\begin{displaymath}
	P = |\psi(A1, B2) + \psi(A2, B1)|^2.
\end{displaymath}
Det kan være fristende å tro at vi har dekket alle muligheter nå, men det viser seg at kvantemekanikken har en overraskelse til på lur. I tilfellet med identiske partikler må vi også ta hensyn til hva slags identiske partikler det dreier seg om. Vi må skille mellom to klasser bosoner og fermioner. Heliumkjerner tilhører klassen bosoner og gir interferens slik som vist ovenfor. Elektroner og protoner hører derimot til klassen fermioner, og da får interferensen en viktig fortegns-forskjell:
\begin{displaymath}
	P = |\psi(A1, B2) - \psi(A2, B1)|^2.
\end{displaymath}
Siden situasjonen er symmetrisk må 
\begin{displaymath}
	\psi(A1, B2) = \psi(A2, B1)
\end{displaymath}
som betyr at det er null sannsynlighet for at elektronene får slik retning ut fra kollisjonen at de treffer detektorene 1 og 2.

\section{Bølgefunksjonen $\psi$}
\label{sec:kvante:psi}
\emph{(Dette avsnittet er i hovedsak skrevet før de foregående ble lagt til og krever dermed litt omarbeiding for å gi en naturlig flyt i teksten. Dette vil bli gjort på et senere tidspunkt.)} \\
\vspace{12pt}

En kvantemekanisk beskrivelse av er som vi har sett svært annerledes enn det vi er vant med. Det viser seg at hvis du vil forsøke å forutsi hvilket resultat en måling---for eksempel av posisjonen eller farten til et elektron---vil gi må du nøye deg med å forutsi hvordan sannsynligheten for ulike måleresultater du kan få er. Dette dreier seg ikke om den vanlige måleusikkerheten som skyldes at måleinstrumentet ikke er perfekt, men det er en underliggende fysisk realitet. La oss for konkrethet se på posisjonen til et elektron. I klassisk fysikk som vi kjenner den fra før ville vi benevnt posisjonen med en vektor
\begin{displaymath}
	\vec{r} = (x,y,z)
\end{displaymath}
som beskriver posisjonen til elektronet med tre koordinater som i prinsippet kan bestemmes eksakt. Hvis vi kjenner alle kreftene som virker kan vi beregne hvordan posisjonen varierer med tiden,
\begin{displaymath}
	x = x(t),\quad y = y(t),\quad z=z(t),
\end{displaymath}
slik at vi vet nøyaktig hvor elektronet er også på et vilkårlig tidspunkt i fremtiden.

En kvantemekanisk beskrivelse av elektronet innebærer en såkalt bølge\-funksjon som forteller oss om sannsynligheten for å finne elektronet på et bestemt sted. I det videre vil jeg begrense meg til å beskrive endimensjonal bevegelse for å forenkle notasjonen, men konseptet kan enkelt utvides til tre dimensjoner. Til å beskrive elektronets posisjon bruker vi funksjonen $\psi(x)$, men denne må tolkes på en helt annen måte enn vektoren $\vec{r}$ ovenfor. For det første vil $\psi(x)$ normalt ha en verdi som er ulik null i mer enn ett punkt, noe som betyr at elektronet ikke har en fast definert posisjon, men kan tenkes å være flere steder. Merk at dette faktisk ikke er et uttrykk for at det finnes informasjon om hvor elektronet egentlig er som vi mangler, men at posisjonen til elektronet ikke er fast definert\footnote{Denne bemerkningen fortjener egentlig en lang diskusjon, men dette er ikke stedet for denne diskusjonen. Det er skrevet mye om dette andre steder.}. Figur \ref{fig:kvante:psix} viser to mulige funksjoner $\psi(x)$ som beskriver posisjonen til et elektron som befinner seg ved eller nær $x=1~\mathrm{m}$. Den stiplete blå linjen er kun ulik null i et relativt lite område nær $x=1~\mathrm{m}$. Dette betyr at vi har en ganske stor visshet om hvor elektronet som beskrives av denne $\psi$-funskjonen er. Den heltrukne røde linjen er mer fordelt utover og innebærer derfor at det er et større område der det er sannsynlig at vi vil finne elektronet. 
\begin{figure}[htp]
\begin{center}
	\includegraphics[width=.5\textwidth]{./psix}
	\caption{}
	\label{fig:kvante:psix}
	\end{center}
\end{figure}

Hva skjer så når vi prøver å måle hvor elektronet er? Da vil vi---uansett hvor konsentrert eller utspredt bølgefunksjonen som beskriver det er---finne elektronet lokalisert i ett punkt\footnote{Selvfølgelig med en presisjon som er begrenset av måleinstrumentet vi bruker.}. Dette innebærer at når vi måler hvor elektronet er endrer vi samtidig bølgefunksjonen som beskriver elektronet til å bli mye mer konsentrert rundt ett punkt, men dette punktet trenger ikke nødvendigvis å være der den tidligere hadde maksimum. Det vil imidlertid alltid være et sted der $\psi(x)$ før målingen var ulik null. Hvis vi kjenner bølgefunksjonen før vi måler kan vi bruke den til å beregne sannsynligheten for å finne elektronet i ulike områder. Det gjør vi ved å integrere over kvadratet av absoluttverdien av $\psi(x)$. For eksemel er sannsynligheten for å finne elektronet et sted mellom $x=0,75~\mathrm{m}$ og $x=0,76~\mathrm{m}$
\begin{equation}
	P ( 0,75~\mathrm{m}<x<0,76~\mathrm{m}) = \int_{0,75~\mathrm{m}}^{0,76~\mathrm{m}}|\psi(x)|^2\d x
	\label{eq:kvante:prob}
\end{equation}
Noen bemerkninger om funksjonen $\psi(x)$
\begin{enumerate}
\item
For at integralet i ligning (\ref{eq:kvante:prob}) skal gi en meningsfull sannsynlighet må funksjonen $\psi(x)$ oppfylle et normaliseringskrav, nemlig 
\begin{displaymath}
	1 = \int_{-\infty}^\infty |\psi(x)|^2\d x.
\end{displaymath}
Dette betyr for det første at sannsynligheten for å finne elektronet et eller annet sted er 1. For det andre, siden $|\psi(x)|^2 \geq 0$ overalt vil vi da være sikret å finne sannsynlighet $P\leq1$ dersom vi integrerer over et kortere intervall.
\item
Når vi integrerer over kvadratet av absoluttverdien til funksjonen ($|\psi|^2$) og ikke bare over kvadratet av funksjonen ($\psi^2$) er det fordi det viser seg at vi generelt må tillate $\psi(x)$ å ha komplekse verdier. I enkelte sammenhenger er dette av stor betydning, men i denne teksten får vi ikke behov for å studere dette videre.
\item
Vi har så langt diskutert bølgefunksjonen som en helt vanlig funksjon som tar inn ett tall (posisjonen) og gir ut ett tall (som riktignok kan være komplekst) som kan relateres til sannynligheten for å finne elektronet akkurat der. Generelt vil vi trenge litt mer kompliserte konstruksjoner. En første enkel utvidelse av dette er å ta med de to andre romlige koordinatene samt å ta med en tidsavhengighet som beskriver at bølgefunksjonen endrer seg når tiden går\footnote{En vanlig konvensjon er å bruke stor $\Psi$ for å betegne en bølgefunksjon som avhenger av både rom- og tidskoordinater, mens man bruker liten $\psi$ dersom bølgefunksjonen kun avhenger av de romlige koordinatene, men ikke endrer seg når tiden går.}:
\begin{displaymath}
	\Psi(x,y,z,t).
\end{displaymath}
Videre er det i en del sammenhenger---blant annet en vi skal se på snart og få mye bruk for i resten av denne teksten---ofte nyttig å la bølgefunksjonen gi ut en vektor i stedet for bare et tall:
\begin{displaymath}
	\Psi(x,y,z,t) = \left[ \begin{array}{cc}\phi_1(x,y,z,t) \\ \phi_2(x,y,z,t) \end{array}\right],
\end{displaymath}
der $\phi_1(x,y,z,t)$ og $\phi_2(x,y,z,t)$ er vanlige funksjoner som gir ut (muligens komplekse) tall.
\end{enumerate}

\section{Schr{\"o}dingerligningen}
\label{sec:kvante:schrodinger}
Gitt en bølgefunksjon $\Psi(x,y,z,t)$ og en funksjon $V(x,y,z)$ som beskriver den potensielle energien som funksjon av posisjonen kan vi finne tidsutviklingen til bølgefunksjonen ved å løse en partiell differensialligning som er kjent som Schr{\"o}dingerligningen (her skrevet opp med bare \'en romlig koordinat):
\begin{displaymath}
	i\hbar\frac{\partial}{\partial t}\Psi(x,t) = \left[ - \frac{\hbar^2}{2m}\frac{\partial^2}{\partial x^2} + V(x)\right]\Psi(x,t)
\end{displaymath}
Her er $m$ massen til partikkelen som beskrives, i vårt tilfelle elektronmassen, og $\hbar = \frac{h}{2\pi} = 1,055\times10^{-34}~\mathrm{Js}$ er den reduserte Planck-konstanten. Schr{\"o}dingerligningen spiller omtrent samme rolle i kvantemekanikken som Newtons andre lov ($\Sigma \vec{F} = m\vec{a}$) spiller i klassisk mekanikk: Den tillater oss å beregne hva som vil skje i fremtiden med kunnskap om hvordan tilstanden er nå. Det er imidlertid noen vesentlige forskjeller. For det første er det bare presisjonen i målingene som begrenser presisjonen i forutsigelsen i klassisk mekanikk. I kvantemekanikk kan vi---som allerede diskutert---kun beregne sannsynligheten for ulike måleresultater, ikke forutsi med sikkerhet hva vi vil måle. Så selv om vi skulle kjenne bølgefunksjonen slik den er nå uten usikkerhet, og dermed kunne forutsi hvordan den vil være i fremtiden uten usikkerhet vil vi fremdeles ikke vite hvilket måleresultat vi ender opp med. For det andre kan Newtons andre lov brukes ``baklengs'': Hvis vi måler hva posisjon og fart er nå kan vi bruke det til å regne ut hva posisjon og fart var på et tidligere tidspunkt. Siden en måling påvirker bølgefunksjonen kan vi ikke bruke Schr{\"o}dingerligningen til å finne ut hvordan bølgefunksjonen var på et tidspunkt før vi målte f.eks. posisjonen til elektronet.

I en generell innføring til kvantemekanikk legges det som regel stor vekt på å løse Schr{\"o}dingerligningen for ulike potensialer $V(x)$. Dette er ikke nødvendig for diskusjonen videre i denne teksten, så det problemet vil ikke bli diskutert videre her.

\section{Spinn}
\label{sec:kvante:spinn}
En del subatomære partikler, inkludert elektronet som vi fokuserer mest på her, har en egenskap som heter spinn. Ordet spinn antyder at det er noe som snurrer rundt, men det er ikke tilfellet. Det er snakk om en iboende egenskap i partikkelen. Likevel kan rotasjonen til en snurrebass være en brukbare analogi for \emph{enkelte aspekter} ved spinnet. Spinn er en vektor som altså har både en størrelse og en retning. I snurrebassanalogien svarer størrelsen til hvor fort den roterer, mens retningen svarer til retningen rotasjonsaksen peker. Positiv retning defineres slik at snurrebassen roterer mot klokken når vi ser den fra den positive siden. Det viser seg at spinnet til subatomære partikler alltid er på formen $\frac{a}{2}\hbar$ der verdien av $a$ avhenger av hvilken partikkel det dreier seg om. Partikler der $a$ er et partall slik at spinnet er et heltall multiplisert med $\hbar$ kalles bosoner. Partikler der $a$ er et oddetall kalles fermioner. Elektronet har $a = 1$ slik at spinnet til elektronet har størrelse $\half\hbar$. Ofte skriver vi dette bare som spinn \half, og jeg vil bruke denne konvensjonen i det videre.

I det videre bryter analogien med en snurrebass fullstendig sammen. I stedet for å måle hvilken retning spinnet til et elektron peker er det enklere å måle projeksjonen av spinnet inn på en vilkårlig akse. Hvis vi gjør dette med en snurrebass vil vi finne en verdi av projeksjonen $s$ slik at  $-S < s < +S$, der $S$ er størrelsen til spinnet. $s=+S$ og $s=-S$ svarer til at aksen vi måler langs er enten parallell eller antiparallell med rotasjonsaksen. $s=0$ svarer til at aksen vi måler langs står normalt på rotasjonsaksen. Når vi gjør denne målingen på et elektron finner vi alltid $+\half$ eller $-\half$ uansett hvilken akse vi måler langs. Sett nå at vi preparerer et elektron med spinnet sitt rettet langs den positive $x$-aksen. Hvis vi nå velger å måle projeksjonen av spinnet på $x$-aksen vil vi med sikkerhet ende opp med resultatet $+\half$. Hvis vi derimot velger å måle projeksjonen av spinnet langs positiv $z$-akse (som står normalt på $x$-aksen) vil vi ende opp med å måle enten $+\half$ eller $-\half$ med 50\% sannsynlighet for hver av verdiene. Hvis vi etter å ha målt projeksjonen langs $z$-aksen igjen måler projeksjonen langs $x$-aksen vil vi ikke lenger med sikkerhet måle $+\half$. Derimot vil vi nå bare ha 50\% sannsynlighet for at målingen viser $+\half$, mens det også er 50\% sannsynlighet for å få verdien $-\half$. 

Som en avslutning av denne første diskusjonen av spinntet tar jeg med at hvis vi igjen preparerer elektronet med spinn opplinjert med $x$-aksen og deretter måler projeksjonen av spinnet på en akse som danner en vinkel $\theta$ med $x$-aksen da blir sannsynlighetene for de to mulige resultatene av målingen
\begin{displaymath}
	P\left(+\half\right) = \cos^2\theta, \quad P\left(-\half\right) = 1- \cos^2\theta = \sin^2\theta.
\end{displaymath}

\section{Måleproblemet}
Både i diskusjonen av posisjonen til elektronet i avsnitt \ref{sec:kvante:psi} og retningen til elektronspinnet i avsnitt \ref{sec:kvante:spinn} var det en underliggende observasjon som ikke ble tydelig formulert:
\begin{quote}
Når vi utfører en måling på et kvantemekanisk system kan vi ikke unngå å samtidig påvirke systemet.
\end{quote}
Dette er helt sentralt, og viser en tydelig forskjell på klassisk mekanikk og kvantemekanikk. I klassisk mekanikk ser vi på objekter som er store nok til at vi kan måle størrelser som for eksempel posisjon eller fart uten relevant påvirkning av størrelsen vi ønsker å måle. Vi kan for eksempel måle posisjonen til en ball ved å \emph{se på den} mens vi har en linjal like ved for å definere måleskalaen. For at vi skal kunne se ballen må det skinne lys på den som reflektres inn i øynene våret. Lyset består av små partikler som kalles fotoner som treffer overflaten til ballen før de sendes videre til, blant annet, øynene våre. Når fotonene treffer ballen gir de den en liten dytt, så i prinsippet kan de endre posisjonen til ballen i prosessen. I praksis er imidlertid bevegelsesmengden til fotonene så liten at de ikke gir noen relevant kraftvirkning på ballen. Derfor kan vi jobbe som om måleprosessen ikke i det hele tatt påvirker det vi ønsker å måle. Dette er ikke tilfellet når vi kommer til kvantemekanikk.

Vi fortsetter å bruke elektronet som eksempel. Massen til et elektron er $m_e = 9,1\times10^{-31}~\mathrm{kg}$. Bevegelsesmengden til et foton avhenger av bølgelengden. Om vi ser på et foton omtrent midt i det synlige spekteret ($\lambda = 550~\mathrm{nm}$) har det bevegelsesmengden $p = \frac{h}{\lambda} = 1,2\times10^{-27}~\mathrm{kg\, m/s}$. Hvis vi bruker dette fotonet til å måle hvor elektronet er vil vi altså samtidig gi elektronet en så kraftig dytt at det etterpå vil ha stor fart bort fra det stedet det var. Slik er det med alle målinger i kvantemekanikken---måleprosessen påvirker systemet vi måler på. Og enda verre, jo mer nøyaktig vi prøver å måle, jo mer vil vi ende opp med å påvirke systemet. Generelt kan vi si at hvis vi ikke kan gjøre målingen på en slik måte at påvirkningen på systemet er neglisjerbar så må vi behandle systemet som kvantemekanisk. Hvis vi derimot kan måle på det uten at målingen gir noen relevant påvirkning på systemet kan vi behandle det med vanlig klassisk mekanikk. 

Det finnes noen spesialtilfeller der vi tilsynelatende unslipper dette måle\-problemet, selv når vi jobber med et system som må behandles kvante\-mekanisk. Et viktig eksempel, og det eneste jeg vil se på her, er gjentatte målinger av elektronspinnet. I avsnitt \ref{sec:kvante:spinn} diskuterte jeg hvordan projeksjonen av spinnet på en akse antar tilfeldige verdier. Men hvis vi repeterer gjentatte målinger langs den samme aksen får vi hele tiden samme resultat. Med andre ord, hvis spinnet er opplinjert med $x$-aksen og vi fortsetter å måle projeksjonen av spinnet inn på $x$-aksen vil ikke målingen endre spinnet. Hvis vi derimot velger å måle spinnet langs en annen akse, for eksempel $z$-aksen, vil målingen påvirke spinnet som diskutert ovenfor.

\section{Sammenfiltring}
Så langt har diskusjonen gitt inntrykk av at hvert enkelt elektron---og andre objekter også for den saks skyld---har hver sin bølgefunksjon. Dette er ikke realiteten. I prinsippet må hele universet beskrives av en felles bølgefunksjon som beskriver alt som er i det. Heldigvis kan vi i mange tilfeller bruke en enklere beskrivelse og likevel få en god representasjon av virkeligheten. Situasjonen er ganske lik til klassisk fysikk: Når vi beregner banen til en stein som blir kastet tar vi hensyn til tyngdekraften fra jorden, men ikke tyngdekraften fra solen, månen og de andre planetene i solsystemet. I prinsippet skulle tyngdekraften fra alt annet i universet vært med i beregningen, men tyngdekraften fra jorden er så mye større enn de andre bidragene at vi kun trenger å ta hensyn til denne. Tilsvarende argument tillater oss i mange tilfeller å regne med \'en bølgefunksjon per objekt i kvantemekanikken. Men det finnes også tilfeller der to eller flere objekter må beskrives av en felles bølgefunksjon for å få en riktig beskrivelse. Når dette er tilfellet sier vi at objektene er \emph{sammenfiltret}. 

Som eksempel på sammenfiltring skal vi se på spinnet til to elektroner. Det er mulig å preparere paret av elektroner slik at summen av spinnet deres målt langs en akse er null. Siden hvert av elektronene kun har muligheten $+\half$ og $-\half$ må altså spinnet til de to elektronene peke hver sin vei. Det spesielle med en sammenfiltret tilstand er at det ikke er slik at det ene elektronet har spinn $+\half$ og det andre $-\half$. I hvert fall ikke før vi har målt. Begge elektronene er i en superposisjon mellom begge mulighetene, men straks vi måler spinnet til det ene elektronet vet vi også hva måleresultatet vil bli når vi måler spinnet til det andre. For å understreke hvor uvant dette er kan vi se på et konkret eksempel. Vi preparerer sammenfiltrede paret av elektroner og plasserer dem i hvert sitt laboratorium adskilt med $3000~\mathrm{km}$. Siden det ikke er mulig å sende noe signal raskere enn lyshastigheten vil det ta minst $10~\mathrm{ms}$ å sende et signal fra det ene laboratoriet til det andre. På et avtalt tidspunkt måler man spinnet til det ene elektronet, og $1~\mathrm{ms}$ senere måler man spinnet til det andre elektronet. Selv om elektronene er for langt fra hverandre til å kunne rekke å kommunisere finner man alltid perfekt korrelasjon: det ene har spinn $+\half$ og det andre har spinn $-\half$. 

En fristende forklaring på hvordan elektronene kan ha denne perfekte korrelasjonen uten tid til å kommunisere er at i realiteten hadde det ene elektronet hele tiden spinn $+\half$ og det andre hadde hele tiden spinn $-\half$. Det var bare det at vi ikke hadde nok informasjon til å vite hvilken som hadde hva frem til vi målte. I 1964 publiserte John Steward Bell en artikkel der han beskrev hvordan man kunne teste om dette faktisk var tilfellet~\cite{PhysicsPhysiqueFizika.1.195}. Jeg skal ikke forsøke å gi en fullstendig beskrivelse av testen her, men bare skissere eksperimentet Bell foreslo. I likhet med eksperimentet jeg har skissert ovenfor skal man også her gjentatte ganger måle spinnet til to sammenfiltrede elektroner som er separert slik at de ikke får tid til å sende et signal fra det ene til det andre. I eksempelet ovenfor målte vi alltid spinnet langs den samme aksen. I Bell sitt eksperiment skal vi måle spinnet langs \'en av tre retninger som er separert med $120^\circ$. Hvilken retning som brukes i hver enkelt måling velges tilfeldig i hvert av de to laboratoriene, og beslutningen gjøres så kort tid før målingen blir gjort at det ikke er tid for et lyssignal å gå fra det ene laboratoriet til det andre. Det viser seg da at korrelasjonen mellom måleresultatene fra de to laboratoriene vil bli ulik avhengig av om
\begin{enumerate}
\item
elektronene på forhånd har et fast definert spinn som vi ikke har nok informasjon til å vite (forutsigelsen til klassisk fysikk), eller
\item
elektronene er begge i en superposisjon mellom de to mulig utfallene av målingen helt frem til vi har målt spinnet til minst ett av elektronene (forutsigelsen til kvantemekanikk).
\end{enumerate}
Eksperimentet har blitt utført med mange forskjellige variasjoner\footnote{De fleste eksperimentene ser på polarisasjonen til fotoner i stedte for spinnet til elek\-troner fordi dette er teknisk enklere å jobbe med, men prinsippet er det samme.}, de første gangene av John Clauser og Stuart Freedman\cite{PhysRevLett.28.938} og Alain Aspect, Philippe Grangier og G\'erard Roger \cite{PhysRevLett.47.460}, og det er ikke mye tvil om at det er kvantemekanikken som forutsier resultatet riktig.