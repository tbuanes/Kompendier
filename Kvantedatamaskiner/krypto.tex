\chapter{Kvantekryptografi}

Klassisk kryptografi har en fundamental utfordring: hvordan distribuere krypteringsnøkkelen. Hvis man har en krypteringsnøkkel som er like lang som meldingen er det prinsipielt umulig å bryte koden, men sikkerheten avhenger fremdeles av at man vet at ingen har klart å få tak i nøkkelen. Vet hjelp av kvantemekanisk sammenfiltring kan vi distribuere en nøkkel over et åpent nettverk og forsikre oss om at kun den rette mottakeren har fått nøkkelen. For å se hvordan dette kan gjøres tar vi utgangspunkt i BB84-protokollen som ble funnet opp av Charles Bennett og Gilles Brassard~\cite{bb84}.

Alice ønsker å sende en kodet melding til Bob, men frykter at Eva kan forsøke å fange opp meldingen og dekryptere den. For å hindre Eva i å kunne dekryptere meldingen bruker Alice og Bob sammenfiltrede elektroner til å utveksle kodenøkkelen. Alice preparerer det nødvendige antallet par av av sammenfiltrede elektroner og sender ett fra hvert par til Bob og beholder det andre selv. Både Alice og Bob måler spinnet til alle elektronene ved å på hvert elektron bruke \'en av de to ordnete, ortonormale basisene
\begin{displaymath}
	V = \left\{ \left[\begin{array}{c}1\\0\end{array}\right], \left[\begin{array}{c}1\\0\end{array}\right] \right\} 
	\quad \mathrm{ok} \quad 
	H = \left\{ \left[\begin{array}{c}\ipsqrt\\ \imsqrt\end{array}\right], \left[\begin{array}{c}\ipsqrt\\ \imsqrt\end{array}\right] \right\}.
\end{displaymath}

