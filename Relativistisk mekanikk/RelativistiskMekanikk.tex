\documentclass[a4paper,norsk,12pt]{article}

\usepackage[norsk]{babel}
\usepackage{enumitem}
\usepackage{color}
\usepackage{amsmath}
\usepackage{wrapfig}
\usepackage{graphicx}
\usepackage[utf8]{inputenc}
\usepackage{wasysym}
\usepackage{csquotes}
\def\d{\ensuremath{\text{d}}}
\def\half{\ensuremath{\frac{1}{2}}}
\def\ivec{\ensuremath{\,\hat{\i}}}
\def\jvec{\ensuremath{\,\hat{\j}}}
\def\kvec{\ensuremath{\,\hat{k}}}

\begin{document}
\section*{Bevarte størrelser}
For å forstå hvilke partikler som kan bli laget i en kollisjon eller et henfall trenger vi å vite om bevarte størrelser
\begin{itemize}
\item Energi (NB! relativistisk --- mer om det senere)
	\begin{itemize}
	\item $n \to p^+e^-\bar{\nu}_e$ er mulig fordi nøytronet har større masse (og dermed energi) enn protonet og elektronet til sammen
	\item $p^+ \to n e^+ \nu_e$ er ikke mulig fordi protonet har mindre masse (og dermed energi) enn nøytronet
	\end{itemize}
\item Bevegelsesmengde
	\begin{itemize}
	\item $e^+e^-\to \gamma$ er ikke mulig fordi bevegelsesmengde ikke bevares
	\end{itemize}
\item Spinn
\item Ladning
\item Leptontall
\item Baryontall
\end{itemize}

\section*{Spesiell relativitetsteori}
Grunnpostulater
\begin{displayquote}
\emph{Fysikkens lover er lik i alle inertialsystemer.}
\end{displayquote}
\begin{displayquote}
\emph{Lyshastigheten er den samme i alle inertialsystemer uavhengig av relativ bevegelse i forhold til kilden.}
\end{displayquote}
Vi definerer to koordinatsystemer 
\begin{align*}
	X:&\quad (t,x,y,z) = (t,\vec{x})\\
	X':&\quad (t',x',y',z') = (t',\vec{x}')
\end{align*}
som beveger seg relativt til hverandre slik at posisjonen til origo i $X'$ beskrives som
\begin{displaymath}
	\vec{P}_O = (vt, 0, 0)
\end{displaymath}
av en observatør som bruker koordinatsystemet $X$. Dette betyr at ved tiden $t=0$ overlapper de to koordinatsystemene, og når tiden går ser observatøren i $X$ at $X'$ beveger seg med konstant hastighet i positiv $x$-retning. Videre krever vi at også $t'=0$ i det de to koordinatsystemene overlapper. Naivt sett vil koordinatene i de to systemene da relateres som 
\begin{align*}
	t' &= t \\
	x' &= x - vt\\
	y' &= y\\
	z' &= z
\end{align*}
Hvis vi nå sender et lys-signal fra origo ved tiden $t = t' = 0$ og observatøren i systemet $X$ måler at hastigheten til lyset er
\begin{displaymath}
	c = \frac{x}{t}
\end{displaymath}
så måler observatøren i systemet $X'$ hastigheten
\begin{displaymath}
	c' = \frac{x'}{t'} = \frac{x-vt}{t} = \frac{x}{t} - v = c - v
\end{displaymath}
Dette er i konflikt med postulat nummer to (og Michelson--Morley-eksperimentet) så koordinattransformasjonen må være feil.

For å få $c=c'$ er det opplagt at tiden må måles ulikt i de de to koordinatsystemene, og at denne ulikheten må henge sammen med ulikheten i hvordan $x$-koordinaten måles. Vi kan derfor prøve en alternativ koordinat-transformasjon:
\begin{align*}
	t' &= a_1x + a_2t \\
	x' &= b_1x + b_2t \\
	y' &= y \\
	z' &= z
\end{align*}
Siden origo i $X'$ beveger seg med fart $v$ målt fra $X$ må

\end{document}