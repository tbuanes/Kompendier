\documentclass[a4paper,norsk,12pt]{article}

\usepackage[norsk]{babel}
\usepackage{enumitem}
\usepackage{color}
\usepackage{amsmath}
\usepackage{wrapfig}
\usepackage{graphicx}
\usepackage[utf8]{inputenc}
\def\d{\ensuremath{\text{d}}}
\def\half{\ensuremath{\frac{1}{2}}}

\begin{document}
Generell prat om hvordan termodynamikk gjør at vi kan håndtere mangepartikkelsystemer uten å trenge informasjon om den mikroskopiske tilstanden.

\section{Kinetisk gassteori}
\subsection{Temperatur}
Fundamentalt sett er temperaturen til et system et mål på hvor mye energi som er lagret i det. I et fast stoff er energien lagret i form av at atomene vibrerer omkring sin likevektsposisjon, samt i form av bevegelsen til ledningeselektronene hvis det faste stoffet er et metall. Vi skal se mer på faste stoffer i avsnittet om varmekapasitet. I fluider er energien lagret i form av kinetisk energi knyttet til translasjon, rotasjon og vibrasjon. 

Vi begynner med å se på en-atomiske gasser, som for eksempel helium. Siden det ikke er noen binding mellom atomer her er det ikke noen "fjær" som kan lagre energi i form av vibrasjon. Vi skal ikke komme videre inn på dette her, men kvantemekanikken forteller oss at en-atomige gasser heller ikke kan lagre energi i form av rotasjon. Dermed er all energi lagret i form av translasjon, slik at den totale energien til gassen kan skrives som
\begin{displaymath}
	E = \sum_i \half mv_i^2
\end{displaymath}
der vi summerer over alle molekylene i gassen.

For at det skal være meningsfullt å snakke om temperatur må systemet være i termodynamisk likevekt\footnote{Strengt tatt er det tilstrekkelig med en lokal likevekt, slik at vi kan tillate en temperaturgradient gjennom gassen. For enkelhets skyld skal vi anta at hele systemet vi ser på er i likevekt.}. Naivt sett skulle man da forvente at hvert atom i gjennomsnitt har den samme energien, men dette viser seg å være feil. I appendiks \ref{apx:boltzmann} viser vi at når en mono-atomisk gass er i likevekt har atomene en hastighet som beskrives av Maxwell-fordelingen,
\begin{displaymath}
	f(v) = 4\pi\left( \frac{m}{2\pi kT} \right)^{3/2} v^2e^{-mv^2/2kT},
\end{displaymath}
der $m$ er massen hvert atom, $T$ er temperaturen til gassen og $k = 1.38\times10^{-23}~\mathrm{J/K}$ er boltzmann-konstanten. $f(v)$ er sannsynlighetstettheten for farten $v$, det vil si at sannsynligheten for å finne et atom med fart i intervallet $v_1<v<v_2$ er
\begin{displaymath}
	P(v_1 < v < v_2) = \int_{v_1}^{v_2} f(v) dv.
\end{displaymath}
Figur (\ref{fig:Maxwelldist}) viser $f(v)$ for heliumgass ved $T=20^\circ\mathrm{C} = 293~\mathrm{K}$. Her ser vi at det store flertallet av atomer har fart omkring ???, mens en liten andel atomer har langt høyere fart. 


\subsection{Trykk}
Fra jobbmaskin

\section{Faste legemer}
Varmekapasitet

Varmeledning

Termisk utvidelse

\end{document}