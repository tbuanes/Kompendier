\documentclass[a4paper,norsk,12pt]{book}

\usepackage[norsk]{babel}
\usepackage{enumitem}
\usepackage{color}
\usepackage{amsmath}
\usepackage{wrapfig}
\usepackage{graphicx}
\usepackage[utf8]{inputenc}
\usepackage{icomma}
\def\d{\ensuremath{\text{d}}}
\def\half{\ensuremath{\frac{1}{2}}}


\begin{document}
\tableofcontents

\pagebreak
Makroskopiske systemer består av så mange atomer at selv om vi vet hvilke regler som beskriver hvert enkelt atom er det håpløst å få til en beskrivelse basert på denne informasjonen på mikroskala. Problemet er egentlig dobbelt - for det første er det umulig å måle tilstanden (posisjon + bevegelse) til alle atomene, og selv om vi klarte å skaffe denne informasjonen er den nødvendige beregningen uoverkommelig selv med de kraftigste datamaskiner vi kan forestille oss. Heldigvis kan vi komme svært langt i å beskrive systemet kun ut fra makroskopiske variabler som blant annet trykk og temperatur. Det er dette termodynamikk dreier seg om.

Historisk sett var utviklingen litt annerledes enn beskrevet ovenfor. Termodynamikk ble utviklet på et tidspunkt da man ennå ikke hadde kjennskap til at alt er bygget opp av atomer så termodynamikken var ikke utviklet som et alternativ til den mikroskopiske beskrivelsen, men for å beskrive en rekke makroskopiske systemer. Senere begynte man å forstå fysikken på mikroskala, og mange av de termodynamiske lovene man hadde funnet frem til ved observasjoner kunne gis en dypere forklaring.

Mange lærebøker presenterer termodynamikken på en måte som minner om den historiske utviklingen. Jeg tror det er bedre å starte med en kikk på den underliggende, mikroskala-fysikken for å forstå systemet vi skal jobbe med litt først. Etter å ha diskutert grunnlaget for de termodynamiske sammenhengene går jeg videre til det makroskopiske nivået og viser hvordan vi kan beskrive systemet uten å ta med detaljene om hva de enkelte atomene gjør.

\chapter{Kinetisk gassteori}
Her beskriver vi hvordan vi kan forstå makroskopiske størrelser som temperatur, varmekapasitet og trykk i gasser som et resultat av summen av bevegelsene til alle atomene i gassen. I mesteparten av kapittelet vil vi kun diskutere ideelle gass, det vil si en hypotetisk gass der
\begin{itemize}
	\item
	gassmolekylene ikke tar opp noe volum,
	\item
	kollisjon mellom gassmolekyler er perfekt elastisk,
	\item
	all indre energi er i form av translasjon\footnote{Dette forutsetter at gassen er  monoatomisk eller at temperaturen er tilstrekkelig lav. Vi skal likevel etterhvert i noen tilfeller også behandle gasser med fleratomige molekyler som om de var ideelle.}, dvs $U = \half mv^2$.
\end{itemize}
Selv om ideell gass er et hypotetisk konsept er dette en god tilnærming for mange reelle gasser. Derfor er ikke en rent teoretisk disuksjon, men også av praktisk nytte.

I avsnitt \ref{sec:kinetiskgassteori:reellgass} kommer en kort beskrivelse av hvordan resultatene i dette kapittelet må modifiseres for reelle gasser der ideell gass-tilnærmingen ikke er tilfredsstillende.

\section{Temperatur}
\label{sec:kinetiskgassteori:temperatur}
Fundamentalt sett er temperaturen til et system et mål på hvor mye energi som er lagret i det. I et fast stoff er energien lagret i form av at atomene vibrerer omkring sin likevektsposisjon, samt i form av bevegelsen til ledningeselektronene hvis det faste stoffet er et metall. Vi skal se mer på faste stoffer i avsnittet om varmekapasitet. I fluider er energien lagret i form av kinetisk energi knyttet til translasjon, rotasjon og vibrasjon. 

Vi begynner med å se på mono	-atomiske gasser, som for eksempel helium. Siden det ikke er noen binding mellom atomer her er det ikke noen 
''fjær'' som kan lagre energi i form av vibrasjon. Vi skal ikke komme videre inn på dette her, men kvantemekanikken forteller oss at en-atomige gasser heller ikke kan lagre energi i form av rotasjon. Dermed er all energi lagret i form av translasjon, slik at den totale energien til gassen kan skrives som
\begin{displaymath}
	E = \sum_i \half mv_i^2
\end{displaymath}
der vi summerer over alle molekylene i gassen.

For at det skal være meningsfullt å snakke om temperatur må systemet være i termodynamisk likevekt\footnote{Strengt tatt er det tilstrekkelig med en lokal likevekt, slik at vi kan tillate en temperaturgradient gjennom gassen. For enkelhets skyld skal vi anta at hele systemet vi ser på er i likevekt.}. Naivt sett skulle man da forvente at hvert atom i gjennomsnitt har den samme energien, men dette viser seg å være feil. I appendiks \ref{apx:maxwellfordeling} viser vi at når en mono-atomisk gass er i likevekt har atomene en hastighet som beskrives av Maxwell-fordelingen,
\begin{equation}
\label{eq:kinetiskgassteori:fv}	
	f(v) = 4\pi\left( \frac{m}{2\pi kT} \right)^{3/2} v^2e^{-mv^2/2kT},
\end{equation}
der $m$ er massen hvert atom, $T$ er temperaturen til gassen og $k = 1.38\times10^{-23}~\mathrm{J/K}$ er boltzmann-konstanten. $f(v)$ er sannsynlighetstettheten for farten $v$, det vil si at sannsynligheten for å finne et atom med fart i intervallet $v_1<v<v_2$ er
\begin{displaymath}
	P(v_1 < v < v_2) = \int_{v_1}^{v_2} f(v) \d v.
\end{displaymath}
Figur (\ref{fig:kinetiskgassteori:Maxwelldist}) viser $f(v)$ for heliumgass ved $T=20^\circ\mathrm{C} = 293~\mathrm{K}$ og $T = 1000~\mathrm{K}$. Ved $293~\mathrm{K}$ ser vi at det store flertallet av atomer har en fart omkring $1000~\mathrm{m/s}$, mens en del atomer har vesentlig større fart. Når temperaturen øker til $1000~\mathrm{K}$ ser vi to endringer. For det første ser vi at temperaturen flest atomer har nå har økt til omkring $2000~\mathrm{m/s}$. For det andre ser vi at spredingen i farten til atomene er større, slik at en mindre andel av atomene har den hyppigst forekommende farten.

\begin{figure}[htp]
	\includegraphics{./MaxwellDistributionHelium}
	\caption{Sannsynlighetsfordeling for farten til atomene i heliumgass ved temperatur 293 K (blå) og 1000 K (rød).}
	\label{fig:kinetiskgassteori:Maxwelldist}
\end{figure}

For å finne den total energien som er lagret i gassen må vi summere over den kinetiske energien til alle enkeltatomene. Siden vi så langt har antatt at alle atomene har den samme massen gir dette ganske enkelt\footnote{Når vi omtaler energien i gassen som helhet bruker vi symbolet $U$ i stedet for $E$ siden det er det vanligste symbolet brukt på indre energi i litteraturen.}
\begin{displaymath}
	U = N\frac{1}{2}m\langle v^2 \rangle
\end{displaymath}
Der $N$ er antall atomer i gassen og $\langle v^2 \rangle$ er middelverdien av $v^2$. Merk forøvrig at $\langle v^2 \rangle \neq \langle v \rangle^2$ siden fartsfordelingen ikke er symmertisk om den mest sannsynlige farten. Dette diskuteres mer i appendiks \ref{apx:maxwellfordeling}.

Middelverdien til $v^2$ finner vi fra hastighetsfordelingen (\ref{eq:kinetiskgassteori:fv}) ved å beregne integralet
\begin{displaymath}
\begin{aligned}
	\langle v^2 \rangle &= \int_0^\infty v^2 f(v) \d v \\
	&= \int_0^\infty 4\pi\left(\frac{m}{2\pi kT}\right)^{3/2} v^4 e^{-mv^2/2kT}\d v.
\end{aligned}
\end{displaymath}
Før vi forsøker å beregne dette integralet omorganiserer vi faktorene litt og gjør et variabelbytte,
\begin{displaymath}
\begin{aligned}
	\langle v^2 \rangle &= \frac{8kT}{m\sqrt{\pi}} \int_0^\infty \left(\frac{m}{2\pi kT}\right)^{3/2}  e^{-mv^2/2kT}\frac{mv\d v}{2kT} \\
	&=  \frac{8kT}{m\sqrt{\pi}} \int_0^\infty y^4 e^{-y^2} \d y,
\end{aligned}
\end{displaymath}
der $y^2 = mv^2/2kT$, og følgelig $y\d y = mv\d v/2kT$. Integralet er her av en klasse som kalles gaussiske integraler og som kan beregnes relativt enkelt. Vi skal ikke gå nærmere inn på hvordan det beregnes her, men bare konstantere at
\begin{displaymath}
	\int_{-\infty}^\infty y^4e^{-y^2} \d y = \frac{3}{4}\sqrt{\pi}.
\end{displaymath}
Siden integralet vårt kun går fra $0$ til $\infty$, og siden integranden er en jamn funksjon har integralet vårt halve verdien av dette, altså $3\sqrt{\pi}/8$. Vi finner dermed at middelverdien til $v^2$ er
\begin{displaymath}
	\langle v^2 \rangle = \frac{8kT}{m\sqrt{\pi}}\cdot\half\cdot\frac{3}{4}\sqrt{\pi} = 3\frac{kT}{m}.
\end{displaymath}
og dermed blir energien til gassen 
\begin{equation}
\label{eq:kinetiskgassteori:Eideellgass}
	U = N\half m\langle v^2 \rangle = \frac{3}{2}NkT
\end{equation}
Ofte måles stoffmengden i mol i stedet for i antall atomer. Et mol er $6.022~\times10^23$ atomer. Uttrykt i form av mol er energien til gassen 
\begin{equation}
\label{eq:kinetiskgassteori:EideellgassMol}
	U = \frac{3}{2}nRT
\end{equation}
der $n$ er antall mol i gassen og $R = 8.31~\mathrm{J/(mol\cdot K)}$ er en konstant som er relatert til boltzmann-konstanten $k$, og kalles den universelle gasskonstanten.	

Fra det vi har utledet nå ser vi altså at temperaturen til en mono-atomisk gass er direkte proposjonal til summen av kinetisk energi som alle atomene i gassen har. Vi skal ikke se i detalj på gasser med molekyler som består av flere atomer her, men bare konstantere at den samme relasjonen gjelder der også, altså
\begin{displaymath}
	U = N\half m_{CoM}\langle v^2\rangle = \frac{3}{2}NkT.
\end{displaymath}
Merk at det bare er kinetisk energi knyttet til masse-senterets translasjon som går inn i energi-beregningen her, mens energi knyttet til rotasjon eller intern vibrasjon i molekylene ikke er med.


%Siden det er et enormt stort antall atomer kan vi tilnærme summen med et integral der vi bruker fordelingsfunksjonen (\ref{eq:kinetiskgassteori:fv}) som vektfunksjon for å ta hensyn til hvor stor andel av atomene som har enhver gitt fart,
%\begin{displaymath}
%\begin{aligned}
%	E &= \int_0^\infty N vf(v)mv\d v \\
%	&= \int_0^\infty 4\pi N\left(\frac{m}{2\pi kT}\right)^{3/2} mv^4 e^{-mv^2/2kT} \d v.
%\end{aligned}
%\end{displaymath}
%Her er $N$ antall atomer i gassen vi ser på. Denne faktoren må være med fordi $f(v)$ er normalisert slik at den gjelder per atom.
%
%Før vi forsøker å beregne dette integralet omorganiserer vi faktorene litt og gjør et variabelbytte,
%\begin{displaymath}
%\begin{aligned}
%	E &= \frac{4Nm}{\sqrt{\pi}} \int_0^\infty \left(\frac{mv^2}{2kT}\right)^{3/2} e^{-mv^2/2kT} v\d v \\
%	&= \frac{8NkT}{\sqrt{\pi}} \int_0^\infty \left(\frac{mv^2}{2kT}\right)^{3/2} e^{-mv^2/2kT} \frac{mv \d v}{2kT} \\
%	& = \frac{8NkT}{\sqrt{\pi}} \int_0^\infty y^4 e^{-y^2} \d y,
%\end{aligned}
%\end{displaymath}
%der $y^2 = mv^2/2kT$, og følgelig $y\d y = mv\d v/2kT$. Integralet er her av en klasse som kalles gaussiske integraler og som kan beregnes relativt enkelt. Vi skal ikke gå nærmere inn på hvordan det beregnes her, men bare konstantere at
%\begin{displaymath}
%	\int_{-\infty}^\infty y^4e^{-y^2} \d y = \frac{3}{4}\sqrt{\pi}.
%\end{displaymath}
%Siden integralet vårt kun går fra $0$ til $\infty$, og siden integranden er en jamn funksjon har integralet vårt halve verdien av dette, altså $3\sqrt{\pi}/8$. Vi finner dermed at den totale energien er 
%\begin{displaymath}
%	E = \frac{8NkT}{\sqrt{\pi}}\cdot \frac{3\sqrt{\pi}}{8} = 3NkT
%\end{displaymath}



\section{Trykk}
Trykket i et fluid kan forstås som et resultat av bevegelsesmengden til molekylene\footnote{Fluidet består enten av enkeltatomer eller av atomer satt sammen til molekyler. For enkelhets skyld omtaler vi de bare som molekyler her, men diskusjonen er identisk uansett om gassen består av enkeltatomer eller sammensatte molekyler} i fluidet. For å gjøre dette litt mer konkret studerer vi det  trykket fluidet forårsaker på veggene til en beholder. Vi har sett i avsnitt \ref{sec:kinetiskgassteori:temperatur} at molekylene i fluidet beveger seg tilfeldig rundt med en gitt hastighetsfordeling. Denne tilfeldige bevegelsen vil gi stadige kollisjoner med veggene av beholderen. Hver kollisjon kan sees på som et elastisk støt\footnote{Dette er en liten forenkling, men resultatene vi utleder blir likevel riktig så lenge gassen og beholderen er i termisk likevekt.} der hastighetskomponenten til atomet normalt på veggen beholder størrelsen, mens retningen snues. Komponenten parallelt med veggen endres ikke. Dette er illustrert i figur \ref{fig:kinetiskgassteori:deltaV}. Gitt hastighet før kollisjonen
\begin{displaymath}
	\vec{v}_\text{før} = v_\perp \hat{e}_\perp + v_{||} \hat{e}_{||},
\end{displaymath}
der $ \hat{e}_\perp$ og $\hat{e}_{||}$ er enhetsvektorer henholdsvis normalt på veggen og parallelt med veggen, vil altså farten etter kollisjonen være
\begin{displaymath}
	\vec{v}_\text{etter} = -v_\perp \hat{e}_\perp + v_{||} \hat{e}_{||}.
\end{displaymath}
Siden hastighetskomponenten normalt på veggen endrer fortegn får vi en endring i bevegelsesmengde 
\begin{displaymath}
	\Delta \vec{p} = -2mv_\perp\hat{e}_\perp.
\end{displaymath}
Dette betyr at veggen har virket på atomet med en kraft normalt på veggen. Motkraften til denne kraften er den kraften fra molekylet som forsøker å dytte veggen utover. Siden trykk er kraft normalt på en flate delt på arealet av en flate kan vi altså finne trykket ved å summere effekten av alle molekylene som kolliderer med veggen.

\begin{figure}[tp]
\begin{center}
	\includegraphics[height=.4\textwidth]{./DeltaVkollisjon}
	\caption{Vi modellerer kollisjonene mellom gassmolekylene og veggen av beholderen som fullstendig elastisk. Da vil hastighetskomponenten parallelt med veggen være uendret mens hastighetskomponenten normalt på veggen beholder størrelsen, men retningen snur.}
	\label{fig:kinetiskgassteori:deltaV}
\end{center}
\end{figure}

Før vi ser mer på det kvantitative i denne prosessen gjør vi noen viktige observasjoner:
\begin{itemize}
\item
Siden økt temperatur gir økt gjennomsnittshastighet til molekylene vil kollisjonene i gjennomsnitt gi en større kraft mot veggen når temperaturen øker. Dermed øker trykket når temperaturen øker hvis alt annet holdes likt.
\item
Siden økt tetthet gir flere kollisjoner per tidshenhet vil trykket øke når tettheten øker hvis alt annet holdes likt.
\end{itemize}

Når vi nå skal se kvantitativt på sammenhengen mellom molekylbevegelse og trykk skal vi først anta at alle atomene har samme hastighetskomponent $v_\perp$. Dette er ikke riktig, og etterpå skal vi generalisere resultatet til en realistisk hastighetsfordeling.
Vi ser nå på et sylindrisk volum med endeflate $A$ plassert på beholderveggen og lengde $\d\ell = v_\perp \d t$. I løpet av tidsintervallet $\d t$ vil alle molekylene i fluidet som har normalkomponenten av hastigheten sin rettet mot veggen kollidere med veggen og sprette tilbake. Dette gir impulsen
\begin{displaymath}
	I = F_\perp\d t = \half N|\Delta \vec{p}|,
\end{displaymath}
der N er antall molekyler i volumet $V = A\d\ell  = Av_\perp\d t$. Faktoren \half kommer av at med tilfeldig bevegelse vil halvparten av molekylene til enhver tid ha en hastighetskomponent mot veggen, mens andre halvparten har en hastighetskomponent bort fra veggen. Om vi lar $\eta$ være tettheten av molekyler kan vi da regne ut trykket som
\begin{displaymath}
\begin{aligned}
	p = \frac{F_\perp}{A} &= \half\frac{\eta V}{A}\frac{|\Delta \vec{p}|}{\d t} 
	= \half\frac{\eta Av_\perp\d t}{A\d t}2mv_\perp 
	= \eta mv_\perp^2.
\end{aligned}
\end{displaymath}

Nå er det på tide å kvitte oss med forenklingen at $v_\perp$ er den samme for alle molekylene. Dette kan vi enkelt gjøre ved å erstatte $v_\perp^2$ med gjennomsnittsverdien av denne kvadrerte farten, $\langle v_\perp^2\rangle$. Hvis vi ser på hele hastighetsvektoren til et molekyl, $\vec{v} = v_x \hat{e}_x +  v_y \hat{e}_y +  v_z \hat{e}_z$, så er kvadratet av denne
\begin{displaymath}
	v^2 = \vec{v}\cdot\vec{v} = v_x^2 + v_y^2+v_z^2.
\end{displaymath}
Hvis vi i stedet regner ut gjennomsnittsverdien finner vi
\begin{displaymath}
	\langle v^2\rangle =\langle v_x^2\rangle +\langle v_y^2\rangle + \langle v_z^2\rangle.
\end{displaymath}
Siden bevegelsen er tilfeldig har molekylene lik sannsynlighet for å bevege seg i alle retninger og $\langle v_x^2\rangle = \langle v_y^2\rangle = \langle v_z^2\rangle = \frac{1}{3}\langle v^2\rangle$. Om vi nå velger koordinatsystemet vårt slik at $\hat{e}_\perp = \hat{e}_x$, altså slik at $\vec{v}_\perp = \vec{v}_x$ har vi altså
\begin{displaymath}
	\langle v_\perp^2 \rangle =\frac{1}{3}\langle v^2 \rangle.
\end{displaymath}
Innsatt i uttrykket for trykk har vi da
\begin{displaymath}
	p = \eta m\langle v_\perp^2 \rangle = \frac{1}{3}\eta m\langle v^2 \rangle = \frac{2}{3}\eta \left(\frac{1}{2}m\langle v^2 \rangle\right).
\end{displaymath}
Siden $\half m\langle v^2\rangle$ er den gjennomsnittlige kinetiske energien knyttet til molekylenes translasjon kan vi skrive trykket som
\begin{equation}
\label{eq:kinetiskgassteori:pEk}
	p = \frac{2}{3}\eta E_\mathrm{k,tr} = \frac{2}{3}\frac{N}{V}E_\mathrm{k,tr}.
\end{equation}
Fra avsnitt \ref{sec:kinetiskgassteori:temperatur} vet vi at $E_\mathrm{k,tr}$ er direkte proporsjonal til termperaturen i gassen, nemlig $E_\mathrm{k,tr} = \frac{3}{2}kT$. Innsatt i (\ref{eq:kinetiskgassteori:pEk}) gir dette
\begin{equation}
\label{eq:kinetiskgassteori:eos}
	p = \frac{NkT}{V} = \frac{nRT}{V},
\end{equation}
som viser at trykket i en ideell gass er direkte proposjonalt med temperaturen.


\section{Varmekapasitet}
Ikke alle materialer føles like varme selv om de har samme temperatur. For eksempel vil en metallstang ved $10^\circ\mathrm{C}$ føles kald, mens isopor ved samme temperatur føles varm. Tørr luft ved $100^\circ\mathrm{C}$ føles varm, men vi kan fint oppholde oss i den en liten stund. Vanndamp ved samme temperatur forårsaker straks brannskader. Det er to viktige fysiske størrelser som skaper denne forskjellen: varmekapasitet og varmeledningsevne. Her skal vi se på varmekapasiteten til gasser. Varmekapasiteten til faste stoffer ser vi på i avsnitt \ref{sec:fastestoffer:varmekapasitet}, mens varmeledningsevnen til faste stoffer ser vi på i avsnitt \ref{sec:fastestoffer:varmeledningsevne}. Varmeledningsevnen til gasser ser vi ikke på her da den sjelden er av stor interesse siden konveksjon er mye viktigere i det tilfellet, og det er et tema som krever en del fluidmekanikk for å beskrive skikkelig.

Varmekapasiteten er et mål på hvor mye energi som skal til for å øke temperaturen til et stoff; jo større varmekapasiteten er, jo mer energi skal til for å øke temperaturen et gitt antall grader. Det er vanlig å oppgi varmekapasitet enten som energi per kg og grader celsius eller kelvin\footnote{Siden vi kun er interessert i temperaturforskjeller, og siden celsius- og kelvin-skalaene har like stor avstand mellom gradene er det vilkårlig hvilken vi bruker.}, eller som energi per mol og grader celsius eller kelvin. Varmekapasiteten kan dermed defineres som 
\begin{displaymath}
	C = \frac{1}{m}\frac{\d Q}{\d T}\quad\text{eller}\quad
	C = \frac{1}{n}\frac{\d Q}{\d T}.
\end{displaymath}
Her er $Q$ den energien som tilføres som varme til gassen og $T$ temperturen, slik at $\d Q/\d T$ altså er varme per temperatureendring. $m$ er massen til gassen vi ser på og $n$ er stoffmengden, slik at den første versjonen altså har enhet J/(kg$\cdot$K) og den andre J/(mol$\cdot$K).

Før vi sier oss helt fornøyd med definisjonen av varmekapasitet er det imidlertid en detalj til vi må få med oss. Når vi øker temperaturen til en gass øker trykket. Hvis gassen er i en beholder som tillater den å utvide volumet sitt, for eksempel en ballong eller i en sylinder med et bevegelig stempel, vil gassen dermed utvide seg når den varmes opp. Men denne utvidelsen innebærer at gassen gjør et arbeid på omgivelsene sine (f.eks. gummien i ballongen eller stempelet i sylinderen), og dermed brukes en del av energien som ble overført til gassen til å gjøre dette arbeidet. Hvis gassen er i en helt stiv beholder slik at volumet holdes konstant vil gassen ikke gjøre noe arbeid og all den tilførte energien går til å øke temperaturen i gassen. Vi forstår dermed at vi må tilføre mer energi til gassen som får lov til å ekspandere enn til den som holdes ved konstant volum for å få få den samme temperaturendringen. Derfor må vi definere to ulike varmekapasiteter - en for konstant trykk (gassen får utvide seg) og en for kontant volum (gassen får ikke utvide seg):
\begin{displaymath}
	\begin{aligned}
	C_p &= \frac{1}{n}\left.\frac{\d Q}{\d T}\right|_p, \\
	C_V &= \frac{1}{n}\left.\frac{\d Q}{\d T}\right|_V,
	\end{aligned}
\end{displaymath}
og tilsvarende for varmekapasitet med enhet J/(kg$\cdot$K). Fra diskusjonen ovenfor ser vi at vi alltid har
\begin{displaymath}
	C_p > C_V
\end{displaymath}
siden en del av energien går med til å gjøre arbeid på omgivelsene når oppvarmingen skjer ved konstant trykk.

Vi kan også se for oss et tilfelle der gassen får lov til å utvide seg, men ikke like mye som når trykket er konstant. I det tilfellet vil vi få en varmekapasitet som ligger et sted mellom $C_V$ og $C_p$, og som kan skrives som en lineærkombinasjon av disse to varmekapasitetene. Vi skal ikke se videre på denne muligheten i denne teksten.

\subsection{Varmekapasiteten til en monoatomisk, ideell gass}
Vi skal nå beregne varmekapasiteten til monoatomiske ideelle gasser, og vi skal se at vi finner samme varmekapasitet uavhengig av hvor tunge atomer gassen består av. Som nevnt tidligere kan enkeltatomer ikke lagre energi i form av rotasjon, og vi skal også se bort fra muligheten for at elektronene eksiteres siden dette krever mye større energi enn det som er typisk for den termiske energien i gasser. Dermed vil all energi som tilføres gassen bidra til å øke farten til atomene, og dermed den kinetiske energien knyttet til translasjon -- $\half mv^2$. Fra avsnitt \ref{sec:kinetiskgassteori:temperatur} vet vi at indre energi i gassen er 
\begin{displaymath}
	U = \frac{3}{2}NkT = \frac{3}{2}nRT.
\end{displaymath}
Det enkleste tilfellet å regne ut er varmekapsiteten ved konstant volum. Siden gassen verken gjør arbeid eller blir gjort arbeid på vil hele varmen gå til å endre den indre energien. Derfor er 
\begin{displaymath}
	C_V = \frac{1}{n}\left.\frac{\d Q}{\d T}\right|_V = \frac{1}{n}\frac{\d U}{\d T} = \frac{3}{2}R.
\end{displaymath}
Vi ser altså at alle monoatmoiske, ideelle gasser har samme varmekapasitet uansett hva atomvekten til gassen er; og at varmekapasiteten er uavhengig av termperaturen. 

For å beregne varmekapasiteten ved konstant trykk må vi få med effekten av at gassen gjør et arbeid på omgivelsene når en utvider seg, eventuelt at omgivelsene gjør et arbeid på gassen når den komprimeres. Derfor er 
\begin{equation}
\label{eq:kinetiskgassteori:cpw}
	C_P = \frac{1}{n}\left.\frac{\d Q}{\d T}\right|_p = \frac{1}{n}\left(\frac{\d U}{\d T} + \frac{\d W}{\d T}\right), 
\end{equation}
der $\frac{\d W}{\d T}$ er det arbeidet arbeid per temperaturendring i gassen. Fra ligning (\ref{eq:kinetiskgassteori:eos}) har vi
\begin{displaymath}
	pV = NkT = nRT
\end{displaymath}
Om vi differensierer denne ligningen under vilkåret at trykket er konstant, dvs $\d p = 0$, finner vi
\begin{displaymath}
	p\d V = nR\d T.
\end{displaymath}
Venstresiden her er nettopp arbeidet gassen gjør, altså
\begin{displaymath}
	\d W = p\d V = nR\d T.
\end{displaymath}
Om vi setter dette inn i ligning (\ref{eq:kinetiskgassteori:cpw}) finner vi
\begin{displaymath}
	C_P = \frac{1}{n}\frac{\d U}{\d T} + R = C_V + R = \frac{5}{2}R.
\end{displaymath}

\subsection{Varmekapastiteten til fleratomiske, ideelle gasser}
Som nevnt vil vi noen ganger myke opp litt på det siste kravet til ideelle gasser, og tillate at indre energi kan lagres på andre måter enn i form av translasjon. Dette gjør at vi kan behandle gasser med molekyler som består av mer enn et molekyl. Vi vil behandle gassen akkurat som andre ideelle gasser med unntak av at vi tillater at indre energi også kan lagres i form av rotasjon og vibrasjon. Husk at vi bemerket i avsnitt \ref{sec:kinetiskgassteori:temperatur} at det likevel bare er den delen av energien som er i form av translasjonsenergi som påvirker temperaturen til gassen.

Det viser seg at hvis vi forsøker å beskrive rotasjonen og vibrasjonen til gassmolekylene med klassisk mekanikk får vi resultater som er feil. Vi må derfor bruke kvantemeknikk til å beskrive dette. Vi skal ikke går nærmere inn på den kvantemekaniske beskrivelsen her, bare oppsummere resultatene. I klassisk mekanikk kan vi beskrive enhver rotasjon som en kombinasjon av rotasjon om tre ortogonale akser (f.eks. $x, y$ og $z$). Det samme kan vi gjøre i en kvantemekanisk beskrivelse, men det viser seg at ikke alle molekyler kan ha energi knyttet til alle rotasjonsakser. Som tidligere nevnt kan et enkeltatom ikke lagre energi i form av rotasjon i det hele tatt. Hvis vi har et lineær molekyl som, f.eks. O$_2$ eller CO$_2$, kan det lagres energi knyttet til rotasjonen om de to aksene som står normalt på forbindelseslinjen mellom atomene, men ikke om aksen langs forbindelseslinjen. Molekyler som er ikke-lineære kan lagre energi knyttet til rotasojn om tre akser.

Energien som lagres knyttet til hver av de mulige rotasjonsaksene er $E_\text{rot} = \half kT$. Merk forøvrig at dette er nøyaktig den samme energien som er knyttet til translasjon i hver av de tre mulige retningene ($x,y,z$). Det er imidlertid en viktig forskjell her: Kvantemekanikken forteller at det er en minste energi som er nødvendig for å sette molekylet i rotasjon. Med andre ord må temperaturen være tilstrekkelig høy slik at $\half kT$ er større enn denne terskelenergien for at noe energi skal lagres som rotasjon. Hvor stor denne terskelenergien er er avhengig av hvilket molekyl det er snakk om.

Molekyler som består av to eller flere atomer kan også lagre energi ved at de vibrerer. Dette kan man tenke på som om atomene er bundet sammen av en fjær som periodisk trykkes sammen og strekkes. Avhengig av antall atomer og geometrien til molekylet kan det være en eller flere slike vibrasjonsmoder som kan lagre energi. [...]

Antallet tilgjengelige "moder" å lagre energi kaller vi for antall frihetsgrader, vanligvis benevnt med symbolet $\gamma$. Frihetsgradene knyttet til translasjon er alltid tilgjengelig for alle gasser\footnote{Det er mulig å lage systemer som oppfører seg som om de var \'en- eller to-dimensjonale, men det ser vi bort fra her.} vil vi ha $\gamma\leq3$. For mono-atomiske gasser er $\gamma = 3$, og det samme tallet finner vi for fleratomiske gasser når temperaturen er for lav til å aktivere rotasjon- og vibrasjonsmoder. For et to-atomig molekyl ved en temperatur der de kan rotere, men ikke vibrere er $\gamma = 3+ 2 = 5$. For et ikke-lineær molekyl (f.eks. H$_2$O) ved en temperatur der det kan rotere, men ikke vibrere er $\gamma = 3+3 = 6$. Den indre energien til gassen kan nå skrives som 
\begin{displaymath}
	U = \frac{\gamma}{2}nRT.
\end{displaymath}
Vi kan nå som tidligere derivere og finne varmekapasiteten:
\begin{displaymath}
\begin{aligned}
	C_V &= \frac{1}{n}\left.\frac{\d Q}{\d T}\right|_V = \frac{1}{n} \frac{\d U}{\d T} = \frac{\gamma}{2}R \\
	C_p &= \frac{1}{n}\left.\frac{\d Q}{\d T}\right|_p = \frac{1}{n} \frac{\d U}{\d T} + R = C_V + R = \frac{\gamma+2}{2}R
\end{aligned}
\end{displaymath}
Merk at for fleratomige gasser er $\gamma$ temperaturavhengig, slik at for fleratomige gasser er ikke varmekapasiteten uavhengig av temperaturen, men plottet mot stigende temperatur finner vi den som en rekke flate platåer med økende høyde. Ved romtemperatur kan vi regne med at rotasjonsmodene er aktivert, men ikke vibrasjonsmodene. 

\section{Reelle gasser}
\label{sec:kinetiskgassteori:reellgass}


\chapter{Faste legemer}
Varmekapasitet

Varmeledning

Termisk utvidelse

\appendix
\chapter{Maxwellfordelingen}
\label{apx:maxwellfordeling}
Vi skal her utlede Maxwell-fordelingen, altså den fordelingen som beskriver hvilken fart atomene har i en gass med en gitt temperatur. For å utlede hastighetsfordelingen til molekylene i en gass skal vi gjøre to antakelser som forenkler beregningen, og som normalt sett ikke begrenser gyldigheten vesentlig,
\begin{enumerate}
	\item
	Fartsfordelingen er isotropisk; det vil si at det er like stor sannsynlighet for å bevege seg i hvilken som helst retning.
	\item
 	Tettheten er homogen; det vil si at i ethvert volum som er stort nok til at statiske fluktuasjoner i antall gass molekyler kan neglisjeres er tettheten den samme.
\end{enumerate}
Sannsynligheten for at et gitt atom har energi $E_i$ er 
\begin{displaymath}
	P(E_i) = \frac{e^{-E_i/kT}}{\sum_j e^{-E_j/kT}}.
\end{displaymath}
Her skal vi ikke gå inn på hvordan vi kommer frem til denne sannsynligheten, men bare konstantere at eksponensialfaktoren innebærer at det er svært lite sannsynlig at enkeltatomer bærer en stor andel av den totale energien selv om det er mulig. Vi ser videre at faktoren $kT$ setter skalaen for hva som er stor og liten energi. Med andre ord, jo høyrere temperatur ($T$), jo større energi kan vi forvente å finne hos enkeltatomer. 

Siden sammenhengen mellom kinetisk energi og hastighet er $E_k = \half m|\vec{v}|^2$, må sannsynlighet for hastighet $\vec{v} = (v_x,v_y,v_z)$ kunne skrives på formen
\begin{displaymath}
	f(\vec{v}) = C\exp\left(-\frac{m|\vec{v}|^2}{2kT}\right) = C\exp\left(-\frac{m}{2kT}\left(v_x^2+v_y^2+v_z^2\right)\right),
\end{displaymath}
der $C$ er en normaliseringskonstant. For å finne $C$ integrerer vi over alle mulige hastigheter, et integral som må ha verdien 1 siden $f(\vec{v})$ beskriver en sannsynlighetsfordeling. Med andre ord må vi ha
\begin{displaymath}
\begin{aligned}
	\frac{1}{C} &= \int_{-\infty}^{\infty}\int_{-\infty}^{\infty}\int_{-\infty}^{\infty}\exp\left(-\frac{m}{2kT}\left(v_x^2+v_y^2+v_z^2\right)\right)\d v_x \d v_y \d v_z \\
	&= \int_{-\infty}^{\infty}\exp\left(-\frac{mv_x^2}{2kT}\right)\d v_x  \int_{-\infty}^{\infty}\exp\left(-\frac{mv_y^2}{2kT}\right)\d v_y  \int_{-\infty}^{\infty}\exp\left(-\frac{mv_z^2}{2kT}\right)\d v_z \\
	&= \left[ \sqrt{\frac{2kT}{m}}\int_{-\infty}^\infty e^{-s^2} \d s\right]^3, 
\end{aligned}
\end{displaymath}
der $s = v_i\sqrt{m/2kT},~i = {x,y,z}$. Det siste integralet er et gaussisk integral, som har løsningen
\begin{displaymath}
	\int_{-\infty}^\infty e^{-s^2} \d s  = \sqrt{\pi}.
\end{displaymath}
Vi finner altså at normaliseringsfaktoren $C$ har verdien
\begin{displaymath}
	C =  \left(\sqrt{\frac{2kT}{m}}\sqrt{\pi}\right)^{-3} = \left(\frac{m}{2\pi kT}\right)^{3/2},
\end{displaymath}
som gir hastighetsfordelingen
\begin{displaymath}
	f(\vec{v}) = \left(\frac{m}{2\pi kT}\right)^{3/2}\exp\left(-\frac{m|\vec{v}|^2}{2kT}\right).
\end{displaymath}
Hastighetsfordelingen vi har funnet nå er bare et steg på veien til det vi egentlig er interessert i - nemlig fartsfordelingen $f(v)$. Grunnen til at disse to fordelingene er ulike er at det for enhver fart $v$ finnes en rekke kombinasjoner av $v_x, v_y$ og $v_z$ som realiserer denne farten. Om vi nå ser på hastigheten til partikkelen i et tre-dimensjonalt koordinatsystem med $v_x, v_y$ og $v_z$ på aksene vil alle punkter på et kuleskall sentrert i origo og med radius $v = \sqrt{v_x^2+v_y^2 + v_z^2}$ beskrive kombinasjoner som gir farten $v$. Siden arealet til slike kuleskall vokser med størrelsen på $v$ ser vi da straks at jo større farten er, jo flere kombinasjoner er det som gir denne farten. Vi finner nå fartsfordelingen $f(v)$ ved å starte med hastighetsfordelingen $f(\vec{v})$ og
\begin{itemize}
	\item
	erstatte $|\vec{v}|^2$ med $v^2$,
	\item
	multiplisere med vektfaktoren som forteller hvor mange hastighetskombinasjoner som gir en bestemt fart, altså arealet av kuleskallet ($4\pi v^2$).
\end{itemize}
Vi er da fremme ved fartsfordelingen vi ønsket å utlede:
\begin{displaymath}
	f(v) = 4\pi\left(\frac{m}{2\pi kT}\right)^{3/2} v^2 e^{-mv^2/2kT}.
\end{displaymath}



\section{Middelverdier av farten i en Maxwellfordeling}
\label{apx:maxwellfordeling:middel}
Siden Maxwellfordelingen er asymmetrisk vil ulike mål på middelverdien gi ulikt resultat, og det er viktig å bruke det riktige målet i riktig sammenheng. Vi skal her se på tre ulike mål på middelverdien:
\begin{itemize}
	\item
	aritmetisk middel, $\langle v \rangle$,
	\item
	mest sannsylige verdi, dvs.~den farten som flest molekyler har, $v_\text{mp}$,
	\item
	kvadratisk gjennomsnitt, ofte kalt rms (root-mean-square), $v_\text{rms} = \sqrt{\langle v^2 \rangle}$.
\end{itemize}

\subsection{Aritmetisk middel}
Det aritmetiske middelet er det middelmålet som man vanligvis tenker på når man snakker om gjennomsnitt. Dette beregnes vanligvis ved å summere verdien av alle datapunktene (her farten til molekylene) og dele på antallet. I vårt tilfelle er det imidlertid mer praktisk å regne det ut fra fordelingsfunksjonen. Vi finner da det aritmetiske middelet ved å beregne integralet
\begin{displaymath}
	\langle v \rangle = \int_0^\infty v f(v) \d v = 4\pi\left(\frac{m}{2\pi kT}\right)^{3/2} \int_0^\infty v^3 e^{-mv^2/2kT} \d v.
\end{displaymath}
Før vi forsøker å beregne dette integralet omorganiserer vi faktorene litt og gjør et variabelbytte,
\begin{displaymath}
\begin{aligned}
	\langle v \rangle &= \sqrt{\frac{32kT}{m\pi}} \int_0^\infty \sqrt{\frac{2kT}{mv^2}}\left(\frac{mv^2}{2 kT}\right)^{3/2}  e^{-mv^2/2kT}\frac{mv\d v}{2kT} \\
	&= \sqrt{\frac{32kT}{m\pi}} \int_0^\infty \frac{mv^2}{2 kT} e^{-mv^2/2kT}\frac{mv\d v}{2kT} \\
	&=   \sqrt{\frac{32kT}{m\pi}}  \int_0^\infty y^3 e^{-y^2} \d y,
\end{aligned}
\end{displaymath}
der $y^2 = mv^2/2kT$, og følgelig $y\d y = mv\d v/2kT$. Integralet er her av en klasse som kalles gaussiske integraler og som kan beregnes relativt enkelt. Vi skal ikke gå nærmere inn på hvordan det beregnes her, men bare konstantere at
\begin{displaymath}
	\int_0^\infty y^3 e^{-y^2} \d y = \half.
\end{displaymath}
Dermed finner vi at det aritmetiske middelet av farten er
\begin{equation}
	\langle v \rangle = \sqrt{\frac{8kT}{m\pi}}.
\end{equation}


\subsection{Mest sannsynlige verdi}
Den mest sannsynlige verdien til farten gitt ved Maxwellfordelingen finner vi der fordelingsfunksjonen har et topp-punkt. Dette punktet finner vi enkelt ved å sette den deriverte av funksjonen lik 0,
\begin{displaymath}
	0 = \frac{\d}{\d v}f(v) = \frac{\d}{\d v} \left[ 4\pi\left(\frac{m}{2\pi kT}\right)^{3/2} v^2 e^{-mv^2/2kT}\right].
\end{displaymath}
For å forenkle notasjonen litt dropper vi normaliseringskonstanten som uansett ikke har noen innvirkning på hvor topp-punktet ligger,
\begin{displaymath}
\begin{aligned}
	0 &= \frac{\d}{\d v}v^2e^{-mv^2/2kT} \\
	&= 2ve^{-mv^2/2kT} - v^2\left(\frac{mv}{kT}\right)e^{-mv^2/2kT} \\
	&= 2v\left(1-\frac{mv^2}{2kT}\right)e^{-mv^2/2kT}.
\end{aligned}
\end{displaymath}
Vi ser at den deriverte har to nullpunkter. $v=0$ svarer opplagt ikke til funksjonens makspunkt, dermed er det 
\begin{displaymath}
	1-\frac{mv^2}{2kT} = 0
\end{displaymath}
vi er interessert i. Dette forteller oss at den mest sannsynlige hastigheten er
\begin{displaymath}
	v_\text{mp} = \sqrt{\frac{2kT}{m}}.
\end{displaymath}


\subsection{Kvadratisk gjennomsnitt}
For å beregne det kvadratiske gjennomsnittet må vi beregne integralet
\begin{displaymath}
\begin{aligned}
	\langle v^2 \rangle &= \int_0^\infty v^2 f(v) \d v \\
	&= 4\pi\left(\frac{m}{2\pi kT}\right)^{3/2} \int_0^\infty v^4 e^{-mv^2/2kT}\d v.
\end{aligned}
\end{displaymath}
Før vi forsøker å beregne dette integralet omorganiserer vi faktorene litt og gjør et det samme variabelbyttet som vi gjorde da vi beregnet det aritmetiske middelet,
\begin{displaymath}
\begin{aligned}
	\langle v^2 \rangle &= \frac{8kT}{m\sqrt{\pi}} \int_0^\infty \left(\frac{mv^2}{2 kT}\right)^{3/2}  e^{-mv^2/2kT}\frac{mv\d v}{2kT} \\
	&=  \frac{8kT}{m\sqrt{\pi}} \int_0^\infty y^4 e^{-y^2} \d y,
\end{aligned}
\end{displaymath}
der $y^2 = mv^2/2kT$, og følgelig $y\d y = mv\d v/2kT$. Også dette er et gaussisk integral, og heller ikke her skal vi gå nærmere inn på hvordan det beregnes her, men bare konstantere at
\begin{displaymath}
	\int_{-\infty}^\infty y^4e^{-y^2} \d y = \frac{3}{4}\sqrt{\pi}.
\end{displaymath}
Siden integralet vårt kun går fra $0$ til $\infty$, og siden integranden er en jamn funksjon har integralet vårt halve verdien av dette, altså $3\sqrt{\pi}/8$. Vi finner dermed at middelverdien til $v^2$ er
\begin{displaymath}
	\langle v^2 \rangle = \frac{8kT}{m\sqrt{\pi}}\cdot\half\cdot\frac{3}{4}\sqrt{\pi} = 3\frac{kT}{m}.
\end{displaymath}
Det kvadratiske gjennomsnittet finner vi nå enkelt ved å ta kvadratroten av $\langle v^2 \rangle$:
\begin{equation}
	v_\text{rms} = \sqrt{\frac{3kT}{m}}.
\end{equation}
$v_\text{rms}$ er av spesielt stor interesse for oss siden den kinetiske energien til atomene er proposjonal med $v^2$. Dette gjør at $v_\text{rms}$ i de fleste sammenhenger er det mest interessante middelmålet for en Maxwellfordeling.


\subsection{Sammenligning av de ulike middelverdiene}
Vi har nå regnet ut tre ulike mål for middelverdien til farten av molekylene. Alle målene har formen
\begin{displaymath}
	v_\text{middel} = A\sqrt{\frac{kT}{m}}
\end{displaymath}
der $A$ er en konstant som er ulik avhengig av om vi ser på å aritmetisk middel, mest sannsynlige verdi eller kvadratisk middel. Det relative forholdet mellom disse tre målene for middelverdi er altså uavhengig av hvilken gass det er snakk om  og hva temperaturen er. Om vi rangerer de ulike målene fra minst til størst finner vi
\begin{displaymath}
	v_\text{mp} < \langle v \rangle < v_\text{rms}.
\end{displaymath}
At både $\langle v \rangle$ og $v_\text{rms}$ er større enn den mest sannsynlige farten er lett å forstå siden fordelingen strekker seg lengre ut mot stor verdi av farten enn liten. Dermed har vi noen relativt få atomer som trekker opp middelverdien. 

\end{document}