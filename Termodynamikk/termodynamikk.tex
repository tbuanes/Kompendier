\documentclass[a4paper,norsk,12pt]{article}

\usepackage[norsk]{babel}
\usepackage{enumitem}
\usepackage{color}
\usepackage{amsmath}
\usepackage{wrapfig}
\usepackage{graphicx}
\usepackage[utf8]{inputenc}
\def\d{\ensuremath{\text{d}}}
\def\half{\ensuremath{\frac{1}{2}}}

\begin{document}
Generell prat om hvordan termodynamikk gjør at vi kan håndtere mangepartikkelsystemer uten å trenge informasjon om den mikroskopiske tilstanden.

\section{Kinetisk gassteori}
\subsection{Temperatur}
Fundamentalt sett er temperaturen til et system et mål på hvor mye energi som er lagret i det. I et fast stoff er energien lagret i form av at atomene vibrerer omkring sin likevektsposisjon, samt i form av bevegelsen til ledningeselektronene hvis det faste stoffet er et metall. Vi skal se mer på faste stoffer i avsnittet om varmekapasitet. I fluider er energien lagret i form av kinetisk energi knyttet til translasjon, rotasjon og vibrasjon. 

Vi begynner med å se på en-atomiske gasser, som for eksempel helium. Siden det ikke er noen binding mellom atomer her er det ikke noen "fjær" som kan lagre energi i form av vibrasjon. Vi skal ikke komme videre inn på dette her, men kvantemekanikken forteller oss at en-atomige gasser heller ikke kan lagre energi i form av rotasjon. Dermed er all energi lagret i form av translasjon, slik at den totale energien til gassen kan skrives som
\begin{displaymath}
	E = \sum_i \half mv_i^2
\end{displaymath}
der vi summerer over alle molekylene i gassen.

\subsection{Trykk}
Fra jobbmaskin

\section{Faste legemer}
Varmekapasitet

Varmeledning

Termisk utvidelse

\end{document}