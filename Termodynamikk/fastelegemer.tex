\chapter{Faste stoffer}
De fleste faste stoffer er bygget opp av atomer plassert i en periodisk krystallstruktur. Detaljene i denne strukturen er avgjørende for egenskapene til materialet, og vi skal se på noen slike egenskaper her. Det finnes også noen faste stoffer som ikke har en slik periodisk krystallstruktur, f.eks. glass der atomene har tilfeldige plasseringer. Biologiske materialer som tre har en mer komplisert struktur. I dette avsnittet begrenser vi oss til å se på materialer med en fast krystallstruktur.

Når vi skal se på egenskapene til faste stoffer må vi skille mellom isolatorer og ledere/metaller.\footnote{I tillegg har vi også halvledere som har mest til felles med isolatorer, men som likevel har en del interessante egenskaper vi ikke ser hos isolatorene. Dette blir imidlertid ikke diskutert her.} I et metall er noen av elektronene så løst bundet at de essensielt sett kan bevege seg fritt i hele metallstykket. Det er dette som gjør at metaller kan lede elektrisk strøm, og det gjør også at elektronene kan gi et viktig bidrag til metallets varmeledningsevne. I isolatorer derimot er alle elektronene tett bundet til atomkjernene. Dermed kan ikke elektronene bidra til transport av verken elektrisk strøm eller varme slik som i metaller, og de termiske egenskapene bestemmes utelukkende av hvordan atomene i krystallstrukturen vibrerer.

\section{Termisk utvidelse}
Tilnærmet alle\footnote{Et viktig unntak her er vann i temperaturområdet 0 til $4^\circ~\mathrm{C}$.} materialer utvider seg når de varmes opp og trekker seg sammen når de kjøles ned. Vi kan  utvidelsen av et krystalinsk stoff kvalitativt ved å se på atomenes mikroskopiske bevegelse. Vi modellerer krystallen som en samling av punktmasser som er holdt på plass av et nettverk av fjærer mellom dem, se figur \ref{fig:fastelegemer:krystall}. Punktmassene representerer atomene, mens fjærene representerer de elektriske kreftene som virker mellom dem. 

Vi innser at denne konfigurasjonen gir en likevektsposisjon for hvert enkelt atom der summen av krefter som virker på det er lik 0. Hvis atomet flyttes litt bort fra likevektsposisjonen vil det virke krefter på det som trekker det mot likevektsposisjonen. Hvis vi antar at kreftene mellom atomene oppfører seg som ideelle fjærer gir disse kreftene opphav til et harmonisk potensial som vist i figur \ref{fig:fastelegemer:harmonisk}. Så lenge atomet har energi $> 0$ vil det vibrere om likevektsposisjonen sin med en amplitude som er avhengig av hvor stor energi\footnote{Kvantemekanikken forteller oss at atomet aldri kan ha 0 energi, og at energien bare kan være en av et diskret sett verdier. Dette er ikke viktig for denne diskusjonen, så vi ser ikke mer på det her.} det har---og derfor avhengig av temperaturen. 
I denne modellen ser vi at når temperaturen øker vil atomene kunne bevege seg lengre vekk fra likevektspunktet sitt, men den gjennomsnittlige avstanden mellom atomene forblir den samme. Dette ville ikke gi noen målbar termisk utvidelse, så denne modellen er opplagt ikke riktig. Der modellen feiler er at den modellere potensialet atomet beveger seg i som symmetrisk om likevektsposisjonen. 

\section{Varmekapasitet}

\section{Varmeledning}