\chapter{Faste stoffer}
De fleste faste stoffer er bygget opp av atomer plassert i en periodisk krystallstruktur. Detaljene i denne strukturen er avgjørende for egenskapene til materialet, og vi skal se på noen slike egenskaper her. Det finnes også noen faste stoffer som ikke har en slik periodisk krystallstruktur, f.eks. glass der atomene har tilfeldige plasseringer. Biologiske materialer som tre har en mer komplisert struktur. I dette avsnittet begrenser vi oss til å se på materialer med en fast krystallstruktur.

Når vi skal se på egenskapene til faste stoffer må vi skille mellom isolatorer og ledere/metaller.\footnote{I tillegg har vi også halvledere som har mest til felles med isolatorer, men som likevel har en del interessante egenskaper vi ikke ser hos isolatorene. Dette blir imidlertid ikke diskutert her.} I et metall er noen av elektronene så løst bundet at de essensielt sett kan bevege seg fritt i hele metallstykket. Det er dette som gjør at metaller kan lede elektrisk strøm, og det gjør også at elektronene kan gi et viktig bidrag til metallets varmeledningsevne. I isolatorer derimot er alle elektronene tett bundet til atomkjernene. Dermed kan ikke elektronene bidra til transport av verken elektrisk strøm eller varme slik som i metaller, og de termiske egenskapene bestemmes utelukkende av hvordan atomene i krystallstrukturen vibrerer.

\section{Termisk utvidelse}

\section{Varmekapasitet}

\section{Varmeledning}