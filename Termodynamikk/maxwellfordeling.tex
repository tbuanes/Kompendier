\section{Maxwellfordelingen}
\label{apx:maxwellfordeling}
Vi skal her utlede Maxwell-fordelingen, altså den fordelingen som beskriver hvilken fart atomene har i en gass med en gitt temperatur. 

Utledning av Maxwell-Boltzmann-statistikk. Fra Kårmund sitt kompendium.

Når vi nå skal utlede hastighetsfordelingen til molekylene i en gass skal vi gjøre to antakelser som forenkler beregningen, og som normalt sett ikke begrenser gyldigheten vesentlig,
\begin{enumerate}
	\item
	Fartsfordelingen er isotropisk; det vil si at det er like stor sannsynlighet for å bevege seg i hvilken som helst retning.
	\item
 	Tettheten er homogen; det vil si at i ethvert volum som er stort nok til at statiske fluktuasjoner i antall gass molekyler kan neglisjeres er tettheten den samme.
\end{enumerate}
Siden sannsynligheten for at et gitt atom har energi $E_i$ er 
\begin{displaymath}
	P(E_i) = \frac{e^{-E_i/kT}}{\sum_j e^{-E_j/kT}}
\end{displaymath}
og siden sammenheng mellom kinetisk energi og hastighet er $E_k = \half m|\vec{v}|^2$, må sannsynlighet for hastighet $\vec{v} = (v_x,v_y,v_z)$ kunne skrives på formen
\begin{displaymath}
	f(\vec{v}) = C\exp\left(-\frac{m|\vec{v}|^2}{2kT}\right) = C\exp\left(-\frac{m}{2kT}\left(v_x^2+v_y^2+v_z^2\right)\right),
\end{displaymath}
der $C$ er en normaliseringskonstant. For å finne $C$ integrerer vi over alle mulige hastigheter, et integral som må ha verdien 1 siden $f(\vec{v})$ beskriver en sannsynlighetsfordeling. Med andre ord må vi ha
\begin{displaymath}
\begin{aligned}
	\frac{1}{C} &= \int_{-\infty}^{\infty}\int_{-\infty}^{\infty}\int_{-\infty}^{\infty}\exp\left(-\frac{m}{2kT}\left(v_x^2+v_y^2+v_z^2\right)\right)\d v_x \d v_y \d v_z \\
	&= \int_{-\infty}^{\infty}\exp\left(-\frac{mv_x^2}{2kT}\right)\d v_x  \int_{-\infty}^{\infty}\exp\left(-\frac{mv_y^2}{2kT}\right)\d v_y  \int_{-\infty}^{\infty}\exp\left(-\frac{mv_z^2}{2kT}\right)\d v_z \\
	&= \left[ \sqrt{\frac{2kT}{m}}\int_{-\infty}^\infty e^{-s^2} \d s\right]^3, 
\end{aligned}
\end{displaymath}
der $s = v_i\sqrt{m/2kT},~i = {x,y,z}$. Det siste integralet er et gaussisk integral, som har løsningen
\begin{displaymath}
	\int_{-\infty}^\infty e^{-s^2} \d s  = \sqrt{\pi}.
\end{displaymath}
Vi finner altså at normaliseringsfaktoren $C$ har verdien
\begin{displaymath}
	C =  \left(\sqrt{\frac{2kT}{m}}\sqrt{\pi}\right)^{-3} = \left(\frac{m}{2\pi kT}\right)^{3/2},
\end{displaymath}
som gir hastighetsfordelingen
\begin{displaymath}
	f(\vec{v}) = \left(\frac{m}{2\pi kT}\right)^{3/2}\exp\left(-\frac{m|\vec{v}|^2}{2kT}\right) 
\end{displaymath}
Hastighetsfordelingen vi har funnet nå er bare et steg på veien til det vi egentlig er interessert i - nemlig fartsfordelingen $f(v)$. Grunnen til at disse to fordelingene er ulike er at det for enhver fart $v$ finnes en rekke kombinasjoner av $v_x, v_y$ og $v_z$ som realiserer denne farten. 