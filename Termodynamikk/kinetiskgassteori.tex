\section{Kinetisk gassteori}
Her beskriver vi hvordan vi kan forstå makroskopiske størrelser som temperatur, varmekapasitet og trykk i gasser som et resultat av summen av bevegelsene til alle atomene i gassen. I mesteparten av kapittelet vil vi kun diskutere ideelle gass, det vil si en hypotetisk gass der
\begin{itemize}
	\item
	gassmolekylene ikke tar opp noe volum,
	\item
	kollisjon mellom gassmolekyler er perfekt elastisk,
	\item
	all indre energi er i form av translasjon\footnote{Dette forutsetter at gassen er  monoatomisk eller at temperaturen er tilstrekkelig lav. Vi skal likevel etterhvert i noen tilfeller også behandle gasser med fleratomige molekyler som om de var ideelle.}, dvs $U = \half mv^2$.
\end{itemize}
Selv om ideell gass er et hypotetisk konsept er dette en god tilnærming for mange reelle gasser. Derfor er ikke en rent teoretisk disuksjon, men også av praktisk nytte.

I avsnitt \ref{sec:kinetiskgassteori:reellgass} kommer en kort beskrivelse av hvordan resultatene i dette kapittelet må modifiseres for reelle gasser der ideell gass-tilnærmingen ikke er tilfredsstillende.

\subsection{Temperatur}
\label{sec:kinetiskgassteori:temperatur}
Fundamentalt sett er temperaturen til et system et mål på hvor mye energi som er lagret i det. I et fast stoff er energien lagret i form av at atomene vibrerer omkring sin likevektsposisjon, samt i form av bevegelsen til ledningeselektronene hvis det faste stoffet er et metall. Vi skal se mer på faste stoffer i avsnittet om varmekapasitet. I fluider er energien lagret i form av kinetisk energi knyttet til translasjon, rotasjon og vibrasjon. 

Vi begynner med å se på en-atomiske gasser, som for eksempel helium. Siden det ikke er noen binding mellom atomer her er det ikke noen 
''fjær'' som kan lagre energi i form av vibrasjon. Vi skal ikke komme videre inn på dette her, men kvantemekanikken forteller oss at en-atomige gasser heller ikke kan lagre energi i form av rotasjon. Dermed er all energi lagret i form av translasjon, slik at den totale energien til gassen kan skrives som
\begin{displaymath}
	E = \sum_i \half mv_i^2
\end{displaymath}
der vi summerer over alle molekylene i gassen.

For at det skal være meningsfullt å snakke om temperatur må systemet være i termodynamisk likevekt\footnote{Strengt tatt er det tilstrekkelig med en lokal likevekt, slik at vi kan tillate en temperaturgradient gjennom gassen. For enkelhets skyld skal vi anta at hele systemet vi ser på er i likevekt.}. Naivt sett skulle man da forvente at hvert atom i gjennomsnitt har den samme energien, men dette viser seg å være feil. I appendiks \ref{apx:maxwellfordeling} viser vi at når en mono-atomisk gass er i likevekt har atomene en hastighet som beskrives av Maxwell-fordelingen,
\begin{displaymath}
	f(v) = 4\pi\left( \frac{m}{2\pi kT} \right)^{3/2} v^2e^{-mv^2/2kT},
\end{displaymath}
der $m$ er massen hvert atom, $T$ er temperaturen til gassen og $k = 1.38\times10^{-23}~\mathrm{J/K}$ er boltzmann-konstanten. $f(v)$ er sannsynlighetstettheten for farten $v$, det vil si at sannsynligheten for å finne et atom med fart i intervallet $v_1<v<v_2$ er
\begin{displaymath}
	P(v_1 < v < v_2) = \int_{v_1}^{v_2} f(v) dv.
\end{displaymath}
Figur (\ref{fig:Maxwelldist}) viser $f(v)$ for heliumgass ved $T=20^\circ\mathrm{C} = 293~\mathrm{K}$. Her ser vi at det store flertallet av atomer har fart omkring ???, mens en liten andel atomer har langt høyere fart. 



\subsection{Trykk}
Trykket i et fluid kan forstås som et resultat av bevegelsesmengden til molekylene\footnote{Fluidet består enten av enkeltatomer eller av atomer satt sammen til molekyler. For enkelhets skyld omtaler vi de bare som molekyler her, men diskusjonen er identisk uansett om gassen består av enkeltatomer eller sammensatte molekyler} i fluidet. For å gjøre dette litt mer konkret studerer vi det  trykket fluidet forårsaker på veggene til en beholder. Vi har sett i avsnitt \ref{sec:kinetiskgassteori:temperatur} at molekylene i fluidet beveger seg tilfeldig rundt med en gitt hastighetsfordeling. Denne tilfeldige bevegelsen vil gi stadige kollisjoner med veggene av beholderen. Hver kollisjon kan sees på som et elastisk støt\footnote{Dette er en liten forenkling, men resultatene vi utleder blir likevel riktig så lenge gassen og beholderen er i termisk likevekt.} der hastighetskomponenten til atomet normalt på veggen beholder størrelsen, mens retningen snues. Komponenten parallelt med veggen endres ikke. Gitt hastighet før kollisjonen
\begin{displaymath}
	\vec{v}_\text{før} = v_\perp \hat{e}_\perp + v_{||} \hat{e}_{||},
\end{displaymath}
der $ \hat{e}_\perp$ og $\hat{e}_{||}$ er enhetsvektorer henholdsvis normalt på veggen og parallelt med veggen, vil altså farten etter kollisjonen være
\begin{displaymath}
	\vec{v}_\text{etter} = -v_\perp \hat{e}_\perp + v_{||} \hat{e}_{||}.
\end{displaymath}
Siden hastighetskomponenten normalt på veggen endrer fortegn får vi en endring i bevegelsesmengde 
\begin{displaymath}
	\Delta \vec{p} = -2mv_\perp\hat{e}_\perp.
\end{displaymath}
Dette betyr at veggen har virket på atomet med en kraft normalt på veggen. Motkraften til denne kraften er den kraften fra molekylet som forsøker å dytte veggen utover. Siden trykk er kraft normalt på en flate delt på arealet av en flate kan vi altså finne trykket ved å summere effekten av alle molekylene som kolliderer med veggen.

Før vi ser mer på det kvantitative i denne prosessen gjør vi noen viktige observasjoner:
\begin{itemize}
\item
Siden økt temperatur gir økt gjennomsnittshastighet til molekylene vil kollisjonene i gjennomsnitt gi en større kraft mot veggen når temperaturen øker. Dermed øker trykket når temperaturen øker hvis alt annet holdes likt.
\item
Siden økt tetthet gir flere kollisjoner per tidshenhet vil trykket øke når tettheten øker hvis alt annet holdes likt.
\end{itemize}

Når vi nå skal se kvantitativt på sammenhengen mellom molekylbevegelse og trykk skal vi først anta at alle atomene har samme hastighetskomponent $v_\perp$. Dette er ikke riktig, og etterpå skal vi generalisere resultatet til en realistisk hastighetsfordeling.
Vi ser nå på et sylindrisk volum med endeflate $A$ plassert på beholderveggen og lengde $\d\ell = v_\perp \d t$. I løpet av tidsintervallet $\d t$ vil alle molekylene i fluidet som har normalkomponenten av hastigheten sin rettet mot veggen kollidere med veggen og sprette tilbake. Dette gir impulsen
\begin{displaymath}
	I = F_\perp\d t = \half N|\Delta \vec{p}|,
\end{displaymath}
der N er antall molekyler i volumet $V = A\d\ell  = Av_\perp\d t$. Faktoren \half kommer av at med tilfeldig bevegelse vil halvparten av molekylene til enhver tid ha en hastighetskomponent mot veggen, mens andre halvparten har en hastighetskomponent bort fra veggen. Om vi lar $n$ være tettheten av molekyler kan vi da regne ut trykket som
\begin{displaymath}
\begin{aligned}
	p = \frac{F_\perp}{A} &= \half\frac{nV}{A}\frac{|\Delta \vec{p}|}{\d t} 
	= \half\frac{nAv_\perp\d t}{A\d t}2mv_\perp 
	= nmv_\perp^2.
\end{aligned}
\end{displaymath}

Nå er det på tide å kvitte oss med forenklingen at $v_\perp$ er den samme for alle molekylene. Dette kan vi enkelt gjøre ved å erstatte $v_\perp^2$ med gjennomsnittsverdien av denne kvadrerte farten, $\langle v_\perp^2\rangle$. Hvis vi ser på hele hastighetsvektoren til et molekyl, $\vec{v} = v_x \hat{e}_x +  v_y \hat{e}_y +  v_z \hat{e}_z$, så er kvadratet av denne
\begin{displaymath}
	v^2 = \vec{v}\cdot\vec{v} = v_x^2 + v_y^2+v_z^2.
\end{displaymath}
Hvis vi i stedet regner ut gjennomsnittsverdien finner vi
\begin{displaymath}
	\langle v^2\rangle =\langle v_x^2\rangle +\langle v_y^2\rangle + \langle v_z^2\rangle.
\end{displaymath}
Siden bevegelsen er tilfeldig har molekylene lik sannsynlighet for å bevege seg i alle retninger og $\langle v_x^2\rangle = \langle v_y^2\rangle = \langle v_z^2\rangle = \frac{1}{3}\langle v^2\rangle$. Om vi nå velger koordinatsystemet vårt slik at $\hat{e}_\perp = \hat{e}_x$, altså slik at $\vec{v}_\perp = \vec{v}_x$ har vi altså
\begin{displaymath}
	\langle v_\perp^2 \rangle =\frac{1}{3}\langle v^2 \rangle.
\end{displaymath}
Innsatt i uttrykket for trykk har vi da
\begin{displaymath}
	p = nm\langle v_\perp^2 \rangle = \frac{1}{3}nm\langle v^2 \rangle = \frac{2}{3}n\left(\frac{1}{2}m\langle v^2 \rangle\right).
\end{displaymath}
Siden $\half m\langle v^2\rangle$ er den gjennomsnittlige kinetiske energien knyttet til molekylenes translasjon kan vi skrive trykket som
\begin{equation}
	\label{eq:kinetiskgassteori:pEk}
	p = \frac{2}{3}nE_\mathrm{k,tr} = \frac{2}{3}\frac{N}{V}E_\mathrm{k,tr}.
\end{equation}
Fra avsnitt \ref{sec:kinetiskgassteori:temperatur} vet vi at $E_\mathrm{k,tr}$ er direkte proporsjonal til termperaturen i gassen, nemlig $E_\mathrm{k,tr} = \frac{3}{2}kT$. Innsatt i (\ref{eq:kinetiskgassteori:pEk}) gir dette
\begin{displaymath}
	p = nkT,
\end{displaymath}
som viser at trykket i en ideell gass er direkte proposjonalt med temperaturen.


\subsection{Varmekapasitet}
Ikke alle materialer føles like varme selv om de har samme temperatur. For eksempel vil en metallstang ved $10^\circ\mathrm{C}$ føles kald, mens isopor ved samme temperatur føles varm. Tørr luft ved $100^\circ\mathrm{C}$ føles varm, men vi kan fint oppholde oss i den en liten stund. Vanndamp ved samme temperatur forårsaker straks brannskader. Det er to viktige fysiske størrelser som skaper denne forskjellen: varmekapasitet og varmeledningsevne. Her skal vi se på varmekapasiteten til gasser. Varmekapasiteten til faste stoffer ser vi på i avsnitt \ref{sec:fastestoffer:varmekapasitet}, mens varmeledningsevnen til faste stoffer ser vi på i avsnitt \ref{sec:fastestoffer:varmeledningsevne}. Varmeledningsevnen til gasser ser vi ikke på her da den sjelden er av stor interesse siden konveksjon er mye viktigere i det tilfellet, og det er et tema som krever en del fluidmekanikk for å beskrive skikkelig.

Varmekapasiteten er et mål på hvor mye energi som skal til for å øke temperaturen til et stoff; jo større varmekapasiteten er, jo mer energi skal til for å øke temperaturen et gitt antall grader. Det er vanlig å oppgi varmekapasitet enten som energi per kg og grader celsius eller kelvin\footnote{Siden vi kun er interessert i temperaturforskjeller, og siden celsius- og kelvin-skalaene har like stor avstand mellom gradene er det vilkårlig hvilken vi bruker.}, eller som energi per mol og grader celsius eller kelvin. Varmekapasiteten kan dermed defineres som 
\begin{displaymath}
	C = \frac{1}{m}\frac{\d Q}{\d T}\quad\text{eller}\quad
	C = \frac{1}{n}\frac{\d Q}{\d T}.
\end{displaymath}
Her er $Q$ energien som tilføres og $T$ temperturen, slik at $\d Q/\d T$ altså er energi per temperatureendring. $m$ er massen til gassen vi ser på og $n$ er stoffmengden, slik at den første versjonen altså har enhet J/(kg$\cdot$K) og den andre J/(mol$\cdot$K).

Før vi sier oss helt fornøyd med definisjonen av varmekapasitet er det imidlertid en detalj til vi må få med oss. Når vi øker temperaturen til en gass øker trykket. Hvis gassen er i en beholder som tillater den å utvide volumet sitt, for eksempel en ballong eller i en sylinder med et bevegelig stempel, vil gassen dermed utvide seg når den varmes opp. Men denne utvidelsen innebærer at gassen gjør et arbeid på omgivelsene sine (f.eks. gummien i ballongen eller stempelet i sylinderen), og dermed brukes en del av energien som ble overført til gassen til å gjøre dette arbeidet. Hvis gassen er i en helt stiv beholder slik at volumet holdes konstant vil gassen ikke gjøre noe arbeid og all den tilførte energien går til å øke temperaturen i gassen. Vi forstår dermed at vi må tilføre mer energi til gassen som får lov til å ekspandere enn til den som holdes ved konstant volum for å få få den samme temperaturendringen. Derfor må vi definere to ulike varmekapasiteter - en for konstant trykk (gassen får utvide seg) og en for kontant volum (gassen får ikke utvide seg):
\begin{displaymath}
	\begin{aligned}
	C_p &= \frac{1}{n}\left.\frac{\d Q}{\d T}\right|_p, \\
	C_V &= \frac{1}{n}\left.\frac{\d Q}{\d T}\right|_V,
	\end{aligned}
\end{displaymath}
og tilsvarende for varmekapasitet med enhet J/(kg$\cdot$K). Fra diskusjonen ovenfor ser vi at vi alltid har
\begin{displaymath}
	C_p > C_V
\end{displaymath}
siden en del av energien går med til å gjøre arbeid på omgivelsene når oppvarmingen skjer ved konstant trykk.

Vi kan også se for oss et tilfelle der gassen får lov til å utvide seg, men ikke like mye som når trykket er konstant. I det tilfellet vil vi få en varmekapasitet som ligger et sted mellom $C_V$ og $C_p$, og som kan skrives som en lineærkombinasjon av disse to varmekapasitetene. Vi skal ikke se videre på denne muligheten i denne teksten.

\subsubsection{Varmekapasiteten til en monoatomisk, ideell gass}
Vi skal nå beregne varmekapasiteten til monoatomiske ideelle gasser, og vi skal se at vi finner samme varmekapasitet uavhengig av hvor tunge atomer gassen består av. Som nevnt tidligere kan enkeltatomer ikke lagre energi i form av rotasjon, og vi skal også se bort fra muligheten for at elektronene eksiteres siden dette krever mye større energi enn det som er typisk for den termiske energien i gasser. Dermed vil all energi som tilføres gassen bidra til å øke farten til atomene, og dermed den kinetiske energien knyttet til translasjon - $\half mv^2$. 



\subsection{Reelle gasser}
\label{sec:kinetiskgassteori:reellgass}
