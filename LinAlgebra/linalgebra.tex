\documentclass[a4paper,norsk,12pt]{article}

\usepackage[norsk]{babel}
\usepackage{enumitem}
\usepackage{color}
\usepackage{amsmath}
\usepackage{amssymb}
\usepackage{wrapfig}
\usepackage{graphicx}
\usepackage[utf8]{inputenc}
\usepackage{wasysym}
\usepackage{icomma}
\usepackage{multicol}

\newcommand{\unit}[1]{~\mathrm{#1}}
\newcommand{\ans}[1]{\underline{\underline{#1}}}

\usepackage[svgnames]{xcolor}
\usepackage[most]{tcolorbox}
\usetikzlibrary{shadows}
\newcounter{exa}
\tcbset{
myexample/.style={
  enhanced,
  colback=blue!10!white,
  colframe=black,
  fonttitle=\scshape,
  titlerule=0pt,
  title={\refstepcounter{exa}Eksempel~\theexa},
  title style={fill=blue!40!white},
  coltitle=black,
  drop shadow,
  highlight math style={reset,colback=LightBlue!50!white,colframe=Navy}
  }
}

\tcbset{
mytheorem/.style={
  enhanced,
  colback=DarkGreen!10!white,
  colframe=black,
  fonttitle=\scshape,
  titlerule=0pt,
  title={Teorem},
  title style={fill=DarkGreen!40!white},
  coltitle=black,
  drop shadow,
  highlight math style={reset,colback=DarkGreen!50!white,colframe=DarkGreen}
  }
}


\tcbset{
myproof/.style={
  enhanced,
  colback=DarkGreen!10!white,
  colframe=black,
  fonttitle=\scshape,
  titlerule=0pt,
  title={Bevis},
  title style={fill=DarkGreen!40!white},
  coltitle=black,
  drop shadow,
  highlight math style={reset,colback=DarkGreen!50!white,colframe=DarkGreen}
  }
}

\tcbset{
mydef/.style={
  enhanced,
  colback=red!10!white,
  colframe=black,
  fonttitle=\scshape,
  titlerule=0pt,
  title={Definisjon},
  title style={fill=red!40!white},
  coltitle=black,
  drop shadow,
  highlight math style={reset,colback=red!50!white,colframe=red}
  }
}

\tcbset{
mysummary/.style={
  enhanced,
  colback=yellow!10!white,
  colframe=black,
  fonttitle=\scshape,
  titlerule=0pt,
  title={Oppsummering},
  title style={fill=yellow!40!white},
  coltitle=black,
  drop shadow,
  highlight math style={reset,colback=yellow!50!white,colframe=yellow}
  }
}



\newtcolorbox{texample}{myexample}
\newtcolorbox{ttheorem}{mytheorem}
\newtcolorbox{tproof}{myproof}
\newtcolorbox{tdef}{mydef}
\newtcolorbox{tsummary}{mysummary}

\begin{document}
\section{Lineær algebra}

\subsection{Lineære ligningssett}
Lineære ligninger er ligninger der de ukjente kun forekommer i første potens, og aldri multiplisert med hverandre. For eksempel er ligningene
\begin{displaymath}
	2x + 3 = 5 \quad \text{og}\quad 3 x - y = 4y - 3
\end{displaymath}
lineære, mens
\begin{displaymath}
	x^2 + 3 = 5, \quad xy + 3y = 1 \quad \text{og}\quad x + \frac{3}{y} = 2
\end{displaymath}
ikke er lineære. Matematiske beskrivelser av fysiske systemer ender noen ganger opp med lineære ligninger og andre ganger med ikke-lineære. Det lineære tilfellet er hyppig nok til at det er vel verdt å bruke tid på å finne effektive måter å løse slike ligninger på. Dersom det bare er \'en eller to ukjente er løsningen rask å finne nesten uansett hvordan man griper an problemet, men hvis man har et sett av mange ligninger med mange ukjente er det nødvendig å gå mer systematisk til verks. Selv om det er for systemer med mange ukjente teknikkene vi ser på er her mest relevante kommer eksemplene som blir brukt til å kun ha to eller tre ukjente fordi det er nok til å forstå løsningsstrategien, og få ukjente gjør beregningen mer oversiktelig.

\begin{texample}
Kirchoffs lover $\to$ lineært ligningssett
\end{texample}

\subsection{Geometrisk tolkning av lineære ligninger}
Før vi ser på hvordan vi faktisk skal løse ligningene skal vi se litt på en geometrisk tolkning av ligningene som kan være nyttig for å forstå hva slags løsninger som er mulig å få. Vi begynner med å se på ligningen
\begin{displaymath}
	3x-y = 4y -3
\end{displaymath}
som inneholder de to ukjente $x$ og $y$. En liten omorganisering av ligningen gir oss
\begin{displaymath}
	y = \frac35x + \frac35.
\end{displaymath}
Hvis vi tolker det som et funksjonsuttrykk så ser vi at dette beskriver en rett linje i $xy$-planet. Så lenge vi har en \emph{lineær} ligning må linjen nødvendigvis blir rett. 

Alle punkter på denne linjen oppfyller ligningen---vi har med andre ord uendelig mange løsninger i stedet for \'en unik løsning. Dette er en konsekvens av at vi har to ukjente, men bare en ligning. Dersom vi har en ligning til, for eksempel
\begin{displaymath}
	2y - x = 3,
\end{displaymath}
så har vi nok en rett linje i $xy$-planet. En løsning $(x,y)$ som oppfyller begge ligningen må ligge på begge de to rette linjene. Når vi har to rette linjer i et plan finnes det tre muligheter:
\begin{enumerate}
	\item Linjene skjærer hverandre i ett punkt. Da har vi \emph{en unik løsning} som oppfyller begge ligningene samtidig.
	\item Linjene er parallelle og skjærer hverandre aldri. Da har vi \emph{ingen løsning} som oppfyller begge ligningene samtidig.
	\item Linjene er identisk slik at de ligger oppå hverandre. Da har vi \emph{uendelig mange løsninger} som oppfyller begge ligningene samtidig. 
\end{enumerate}
Flere muligheter enn dette finnes ikke---vi kan aldri få to eller syv løsninger, bare ingen, \'en eller uendelig mange.

Hvis vi går til tre dimensjoner, altså med tre ukjente---$x$, $y$ og $z$---er historien nesten helt lik. En ligning som inneholder alle de tre ukjente, for eksempel
\begin{displaymath}
	2x - 3y + z = 2
\end{displaymath}
beskriver en rett linje gjennom rommet. Dersom ligningen kun inneholder en eller to av de ukjente, for eksempel
\begin{displaymath}
	y = 0\quad\text{eller}\quad 3x-z = 4
\end{displaymath}
beskriver den et plan. 
\end{document}
