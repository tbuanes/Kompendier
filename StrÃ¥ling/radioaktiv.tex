\chapter{Radiokativitet}
De fleste isotoper er stabile---det vil si de endrer ikke karakter med mindre de blir utsatt for en ytre påvirkning som endrer dem. Imidlertid er det noen isotoper som er ustabile. Slike isotoper vil spontant endre seg enten ved å
\begin{itemize}
\item
sende ut en heliumkjerne bestående av 2 protoner og to elektroner ($\alpha$-partikkel),
\item
omgjøre et nøytron til et proton samtidig som det sendes ut et elektron ($\beta$-partikkel),
\item
omgjøre et proton til et nøytron samtidig som det sendes ut et positron ($\beta+$-partikkel),
\item
omgjøre et proton til et elektron ved å fange et av elektronene rundt isotopkjernen ($\beta$-innfanging),
\item
spaltes i to isotopkjerner + et antall nøytroner (fisjon).
\end{itemize}
Enkelte av prosessene etterlater isotopkjernen i en eksitert tilstand. Da vil den etter kort tid henfalle til grunntilstanden samtidig som det sendes ut et høy-energetisk foton ($\gamma$-stråling). Hvilken prosess som skjer og hvor lenge isotopet eksisterer før det skjer avhenger av hvilket isotop vi starter med.

\section{Levetiden til ustabile isotoper}
Henfallet av ustabile isotoper er såvidt vi vet en fullstendig tilfeldig prosess---det vil si at det er ikke på noen måte mulig å forutsi akkurat når henfallet vil skje. Hver enkelt type ustabilt isotop har imidlertid en gjennomsnittlig levetid, $\tau$.  Hvis vi starter med $N_0$ atomer\footnote{Vi antar i denne diskusjonen at antall atomer vi starter med er stort nok til at statistiske fluktuasjoner er neglisjerbare.} vil antallet atomer som ikke har henfalt etter tid $t$ være
\begin{equation}
	N(t) = N_0e^{-t/\tau}.
	\label{eq:radioaktiv:henfall}
\end{equation}
Ofte snakker man om halvveringstiden til de ustabile isotopene i stedet for levetiden. Halvveringstiden $t_{1/2}$ er den tiden det tar før halvparten av isotopene har henfalt. Fra ligning (\ref{eq:radioaktiv:henfall}) finner vi da
\begin{displaymath}
	\half N_0 = N_0e^{-t_{1/2}/\tau}
\end{displaymath}
som viser at sammenhengen mellom levetid og halvveringstid er
\begin{equation}
	t_{1/2} = \tau\ln 2.
\end{equation}
For ulike ustabile isotoper kan halvveringstiden til variere fra små brøkdeler av et sekund til lenger enn universets levetid. 

\section{Kilder til radioaktiv stråling}


\section{$\alpha$-stråling}
$\alpha$-stråling består av heliumkjerner---det vil si atomkjerner med 2 protoner og 2 nøytroner---og de har da elektrisk ladning $+2e$. Den elektriske ladningen gjør at $\alpha$-partiklene kan ionisere atomer. I tillegg vil de---siden de består av protoner og nøytroner---også vekselvirke via den sterke kjernekraften. Kombinasjonen av stor masse (sammenlignet med massen f.eks.~elektroner) og sterk kjernekraft gjør at det kreves lite materiale for å stanse $\alpha$-stråling. {\color{red}[Litt kvantitativt om hva lite materiale betyr]} Den praktiske konsekvensen av dette er at det er enkelt å skjerme seg mot $\alpha$-stråling. Til gjengjeld vil $\alpha$-stråling avsette energien sin i et konsentrert område om den ikke er skjermet, og kan da gi betydelig vev-skade. Dette gjør at radioaktive stoffer som henfaller ved å sende ut $\alpha$-stråling først og fremst er farlig hvis de kommer direkte kontakt med huden eller kommer inn i kroppen. Sistnevnte er spesielt aktuelt med radon-gass som pustes inn.

\section{$\beta$-stråling}
$\beta$-stråling er elektroner eller positroner som sendes ut fra kjernen ved at et nøytron gjøres om til et proton eller omvendt. Ladningen til $\beta$-strålene er $\pm e$. Siden elektroner og positroner ikke kjenner den sterke kjernekraften er $\beta$-stråler vanskeligere å skjerme seg mot enn $\alpha$-stråling, men den lave partikkelmassen gjør at $\beta$-strålene likevel lar seg stoppe relativt enkelt. {\color{red}[Litt kvantitativt om hva lite materiale betyr]}

\section{$\gamma$-stråling}
$\gamma$-stråling er høyenergetiske elektromagnetisk stråling som sendes ut fra atomkjernen. Strålingen sendes ut når en eksitert kjerne går ned til grunntilstanden. Den eksiterte kjernen er som regel et resultat av enten et tidligere $\beta$-henfall eller en elektron-innfanging---det vil si at et av elektronene til atomet fanges inn i kjernen og bidrar til at et proton gjøres om til et elektron. Dette betyr at radioaktive stoffer som er kilde til $\beta$-stråling vanligvis også er kilde til $\gamma$-stråling. Siden fotoner ikke har elektrisk ladning, og ikke merker de sterke kjernekreftene, er $\gamma$-stråling vesentlig vanskeligere å skjerme seg mot enn $\alpha$- og $\beta$-stråling. 

\section{Nøytron-stråling}
Nøytron-stråling regnes vanligvis ikke med blant når man diskuterer radioaktiv stråling, men er likevel en viktig form for partikkelstråling som oppstår i fisjonsprosesser. 

\section{Biologisk effekt av radioaktiv stråling}
Alle typer radioaktiv stråling er i stand til å ionisere atomer og kan derfor skade biologisk vev. Hvor stor skadevirkningen er avhenger både av hvilken type stråling det er, energien til hver enkelt partikkel, og selvfølgelig intensiteten---altså hvor mange partikler man blir truffet av. Når man diskuterer eksponering for radioaktiv stråling er det en rekke ulike fysiske størrelser som er i bruk. En del av de viktigste er oppsummert her \cite{Agashe:2014kda}.

\subsection{Aktivitet}
Aktiviteten måler hvor ofte et radioaktivt henfall skjer, og gir derfor et mål på intensiteten av den radioaktive strålingen. Aktivitet er uavhengig av hva slags type radioaktiv stråling vi har. Måleenheten for aktivitet er bequerel ($1~\mathrm{Bq} = 1~\mathrm{s^{-1}}$). En eldre måleenhet man noen ganger treffer på er Curie ($1~\mathrm{Ci} = 3.7\times10^{10}~\mathrm{Bq}$). Siden aktiviteten måler hvor ofte et radioaktivt henfall skjer avhenger den både av hvilken radioaktiv isotop vi har med å gjøre på grunn av halvveringstiden, og hvor mye vi har av den aktuelle isotopen. 

\subsection{Absorbert dose}
Absorbert dose måler hvor mye energi ioniserende stråling avgir i et volum, normalisert med hvor mye masse som er i dette volumet. Måleenheten for absorbert dose er gray ($1~\mathrm{Gy}= 1~\mathrm{J/kg}$). En eldre enhet man kan treffe på er rad ($1~\mathrm{rad} = 10^{-2}~\mathrm{Gy}$). Siden absorbert dose forteller om størrelsen på energiavsetningen er den nyttig for å beskrive skadepotensialet til den radioaktive strålingen, men den tar ikke hensyn til at ulike typer radioaktiv stråling har ulik biologisk virkning.

\subsection{Ekvivalent dose}
Ekvivalent dose ligner på absorbert dose, men tar også med en vektfaktor som tar hensyn til ulik biologisk virkning av ulik stråling. Enheten for ekvivalen dose er sievert (Sv). Man kan også treffe på enheten rem ($1~\mathrm{rem} = 10^{-2}~\mathrm{Sv}$). Om man eksponeres for en eller flere typer radioaktiv stråling med absorbert dose $D_{T,R}$ av type $R$ i organ $T$ beregnes den ekvivalente dosen som
\begin{equation}
	H_T = \sum_R w_R\times D_{T,R},
\end{equation}
der $w_R$ er en vektfaktoren som tar hensyn til ulik virkning av ulike typer stråling. Tabell \ref{tab:radioaktiv:wR} lister opp vektfaktorene for de ulike strålingstypene.
\begin{table}
\begin{center}
	\begin{tabular}{ll}
	\hline
	\hline
	Strålingstype & $w_R$ \\
	\hline
	Fotoner, elektroner, myoner & 1 \\
	$\begin{aligned}
		\text{Nøytroner, } &E_n < 1~\mathrm{MeV} \\
		&1~\mathrm{MeV}\leq E_n \leq 50~\mathrm{MeV} \\
		&E_n > 50~\mathrm{MeV}
	\end{aligned}$& 
	$\begin{aligned}
		&2.5 + 18.2e^{-(\ln E_n)^2/6} \\
		&5.0 + 17.0e^{-(\ln 2E_n)^2/6} \\
		&2.5 + 3.25e^{-(\ln 0.04E_n)^2/6} \\
	\end{aligned}$\\
	Protoner og ladde pioner & 2 \\
	$\alpha$-partikler og tyngre ioner & 20 \\
	\hline
	\hline
	\end{tabular}
\end{center}
\caption{Vektfaktorer for ulike strålingstyper.}
\label{tab:radioaktiv:wR}
\end{table}

\subsection{Effektiv dose}
Effektiv dose ligner på ekvivalent dose, men tar også med i beregningen at ulike organer har ulik sårbarhet for radioaktiv stråling. Den effektive dosen beregnes som
\begin{equation}
	E = \sum_T w_T\times H_T,
\end{equation}
der $w_T$ ($\sum_T w_T = 1$) er vektfaktoren for organ $T$. I likhet med absorbert dose måles også effektiv dose i sievert.