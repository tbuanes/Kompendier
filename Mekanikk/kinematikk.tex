\chapter{Kinematikk}

Fysikk handler om å observere naturen---enten gjennom å se naturlig forekommende fenomenter eller ved å gjøre eksperimenter---systematisere observasjonene og fra dette lage modeller som lar oss forutsi hva som vil bli utfallet av nye observasjoner. I denne prosessen har det vist seg spesielt nyttig å bruke \emph{matematiske modeller}. For å kunne gjøre det må vi kunne oversette observasjonene til matematikk, og også oversette matematikken tilbake til en beskrivelse av hva vi kan forvente å observere. I dette kapittelet skal vi ta de første skrittene på denne veien ved å se på de sentrale begrepene \emph{posisjon}, \emph{forflytning}, \emph{hastighet/fart} og \emph{akselerasjon}.

\section{Posisjon og forflytning}
For å kunne beskrive posisjonen til et objekt matematisk trenger vi et koordinatsystem. Hvordan vi definerer koordinatsystemet kan vi i utgangspunktet velge helt fritt, men som regel er det nyttigste å bruke et  \emph{kartesisk koordinatsystem}\footnote{Et kartesisk koordinatsystem er et koordinatsystem der alle aksene står vinkelrett på hverandre.} så i første omgang ser vi kun på dette. Siden verden har tre romlige dimensjoner---frem/tilbake, høyre/venstre, opp/ned---trenger vi tre koordinatakser, typisk kalt $x$, $y$ og $z$. 

\begin{tdef}
{\bf Posisjonen} beskriver hvor et objekt er ved hjelp av et koordinatsystem og uttrykkes på formen $(x, y, z) = (a\unit{m},\, b\unit{m},\, c\unit{m})$, der $a$, $b$ og $c$ er tall.  
\end{tdef}

Dersom bevegelsen vi ønsker å studere kun foregår langs en rett linje kan vi klare oss med kun $x$-aksen. Dersom bevegelsen foregår i et plan trenger vi $x$- og $y$-aksen, men ikke $z$-aksen.  For at denne reduksjonen i antall akser vi ser på skal fungere er det viktig å velge et koordinatsystem som er rotert på en fornuftig måte.

\begin{texample}
Vi ser på et objekt som beveger seg langs en rett linje fra posisjon $(0,0,0)$ til posisjon $(3\unit{m},\, 4\unit{m},\, 0\unit{m})$. Siden $z$-komponenten er konstant hele tiden kan vi velge å ikke nevne denne og si at objektet beveger seg fra $(0,0)$ til $(3\unit{m},\,4\unit{m})$. Siden bevegelsen skjer langs en rett linje kan vi forenkle videre, men det forutsetter at vi bruker \emph{et annet koordinatsystem}. Vi lager oss et nytt koordinatsystem med origo samme sted og $x$-aksen\footnote{Vi nevner ikke retningen til $y$- og $z$-aksen siden de uansett ikke er nødvendig for å beskrive bevegelsen.} rettet langs linjen objektet beveger seg. Med det nye koordinatsystemet kan vi si at objektet beveger seg fra posisjon $x=0$ til posisjon $x=5\unit{m}$. 

Merk at avstanden mellom start- og sluttposisjon er lik uavhengig av hvilket koordinatsystem vi bruker til å beskrive bevegelsen. Dette kan vi se ved å regne ut 
\begin{displaymath}
	\sqrt{(3\unit{m}-0\unit{m})^2 + (4\unit{m}-0\unit{m})^2 + (0\unit{m}-0\unit{m})^2} = (5\unit{m}-0\unit{m})
\end{displaymath}
\textcolor{red}{[Sett inn figur]}
\end{texample}
\section{Hastighet og fart}

\section{Akselerasjon}